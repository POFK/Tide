\documentclass[aps,prd,twocolumn,showpacs,superscriptaddress,groupedaddress,nofootinbib]{revtex4}  % for review and submission
%\documentclass[aps,preprint,showpacs,superscriptaddress,groupedaddress]{revtex4}  % for double-spaced preprint
\usepackage{graphicx}  % needed for figures
\usepackage{dcolumn}   % needed for some tables
\usepackage{bm}        % for math
\usepackage{amsmath,amssymb}   % for math
\usepackage{color}
\usepackage{multirow}
%\usepackage{aas_macros}
\hyphenation{ALPGEN}
\hyphenation{EVTGEN}
\hyphenation{PYTHIA}
\newcommand{\mr}{\mathrm}
\newcommand{\mb}{\mathbf}
\renewcommand{\d}{\mathrm{d}}
\newcommand{\e}{\mathrm{e}}
\newcommand{\ii}{\mathrm{i}}
\newcommand{\bea}{\begin{eqnarray}}
\newcommand{\eea}{\end{eqnarray}}
\newcommand{\be}{\begin{equation}}
\newcommand{\ee}{\end{equation}}
\newcommand{\rund}[1]{\left(#1\right)}
\newcommand{\eck}[1]{\left[ #1 \right]}
\newcommand{\bigrund}[1]{\left{#1\right}}
\newcommand{\vc}[1]{\mbox{\boldmath $#1$}}
\newcommand{\dc}{\partial}
\newcommand{\msun}{h^{-1}M_{\odot}}

\def\llabel#1{\label{sc:#1}  {#1}\hspace{0.5cm}}
\def\elabel#1{\label{eq:#1}\fbox{#1}}
%\def\llabel#1{\label{sc:#1}}
%\def\elabel#1{\label{eq:#1}}

% avoids incorrect hyphenation, added Nov/08 by SSR
\hyphenation{ALPGEN}
\hyphenation{EVTGEN}
\hyphenation{PYTHIA}

%=========================================================================
\begin{document}
% The following information is for internal review, please remove them for submission
\widetext
% the following line is for submission, including submission to the arXiv!!
%\hspace{5.2in} \mbox{Fermilab-Pub-04/xxx-E}
%=========================================================================
\title{Cosmic tides in view of halos}
\author{Tian-Xiang Mao}
\affiliation{Key Laboratory for Computational Astrophysics,
National Astronomical Observatories, Chinese Academy of Sciences,
20A Datun Road, Beijing 100012, China}
\affiliation{University of Chinese Academy of Sciences, Beijing 100049, China}

\author{Hong-Ming Zhu}
\affiliation{Key Laboratory for Computational Astrophysics,
National Astronomical Observatories, Chinese Academy of Sciences,
20A Datun Road, Beijing 100012, China}
\affiliation{University of Chinese Academy of Sciences, Beijing 100049, China}

\author{Ue-Li Pen}
\affiliation{Canadian Institute for Theoretical Astrophysics, University of Toronto, 60 St. George Street, Toronto, Ontario M5S 3H8, Canada}
\affiliation{Dunlap Institute for Astronomy and Astrophysics, University of Toronto, 50 St. George Street, Toronto, Ontario M5S 3H8, Canada}
\affiliation{Canadian Institute for Advanced Research, CIFAR Program in Gravitation and Cosmology, Toronto, Ontario M5G 1Z8, Canada}
\affiliation{Perimeter Institute for Theoretical Physics, 31 Caroline Street North, Waterloo, Ontario, N2L 2Y5, Canada}

\author{Yu Yu}
\affiliation{Key laboratory for research in galaxies and cosmology,
Shanghai Astronomical Observatory, Chinese Academy of Sciences,
80 Nandan Road, Shanghai 200030, China}

\author{Xuelei Chen}
\affiliation{Key Laboratory for Computational Astrophysics,
National Astronomical Observatories, Chinese Academy of Sciences,
20A Datun Road, Beijing 100012, China}
\affiliation{University of Chinese Academy of Sciences, Beijing 100049, China}
\affiliation{Center of High Energy Physics, Peking University, Beijing 100871, China}

\author{Jie Wang}
\affiliation{Key Laboratory for Computational Astrophysics,
National Astronomical Observatories, Chinese Academy of Sciences,
20A Datun Road, Beijing 100012, China}
%\affiliation{University of Chinese Academy of Sciences, Beijing 100049, China}

\date{\today}
%=========================================================================
\begin{abstract}
The distribution of halo is a biased tracer of the underlying dark matter density which we can not observe directly. In this paper, we try to reconstruct large scale density field from the distribution of halos. We also study the influence of RSD on the reconstruction.
\end{abstract}

\pacs{}
\maketitle
%\section{\label{sec:level1}First-level heading}
% sections are not used for PRL papers
%=========================================================================
\section{Introduction}
The large-scale structure contains a wealth of information about our Universe.
By measuring large-scale structure we can attempt to
find the answers to such fundamental questions as what the initial
conditions of the Universe are and what its future will be.
{\color{red}There have been a lot of efforts in measuring the large-scale
structure of our Universe (e.g. SDSS,).} 
However, the mode coupling because of nonlinear evolution 
{\color{blue} (here add citation)} and many other observational
problems {\color{red}(such as RSD)} make it difficult to extract cosmological information. 
Additionally, because of the limited volume of surveys, uncertainties 
due to sample variance is also difficult to overcome.
%==================================================================

Dark matter halos, which host observable galaxies and galaxy clusters, 
are the fundamental nonlinear units of cosmic structure. On large scales,
 dark matter halos can be considered as biased tracers 
of the underlying dark matter distribution \citep{1984ApJ...284L...9K}.
Not only the local matter distribution but also
the large-scale environment can affect the shapes, the abundance, 
the accretion, the bias, and the power spectrum of halos
\citep{2016ApJ...825...49C,2012JCAP...03..004S,1996MNRAS.282..347M,2015ApJ...807...37S,2015JCAP...09..028C,2014PhRvD..90j3530L,2014JCAP...05..048C,2013PhRvD..87l3504T}.
{\color{blue}
So the coupling between the large-scale environment and small-scale 
{\color{red}density fluctuation} may be used to extract more information about 
the large-scale structure, which is helpful to suppress the sample variance
of large-scale modes.}
In recent studies, this effect is used to get the 
information of very long-wavelength modes
outside the survey volume by considering the large-scale environment
contribute to the mean density fluctuation in the survey
(or "separate universe")
\citep{2016JCAP...09..007B,2016arXiv160901701C,2016PhRvD..93f3507L,2016JCAP...02..018L, 2015JCAP...08..042W,2003ApJ...585...34M}.
Nevertheless, the isotropic distortions on small scales may derived from
not only the large-scale environment but also some other processes.
In our previous stydies \citep{2012arXiv1202.5804P,2016PhRvD..93j3504Z},
the long-wavelength tidal field is reconstructed by quantifying the local
anisotropy of small-scale density statistics and even a better result
obtained by a developed 3-D method \citep{tides3d:Hu}.
%If we only consider the position of halos, the large-scale density field
%is also can be reconstructed by halos field.
%==================================================================

In this paper, we investigate the cosmic tidal reconstruction in view
of halos.
Different with dark matter density field, reconstruction with halos field has
many advantages. In the dark matter case, we need to take a logarithmic 
transform or map the density fluctuations into a Gaussian distribution to
Gaussianize the smoothed density field or suppress the high weights 
of the high density regions because of quadratic estimator
\citep{2012arXiv1202.5804P,2016PhRvD..93j3504Z,tides3d:Hu}.
However, for halo counts, there is not such problems. Without Gaussianization,
we can quantify the bias of reconstruction easier and get a simpler and 
clearer physical picture, which we will discuss in \ref{sec:performance}. 
Additionally, since most observed tracers of large-scale structure, 
such as galaxies, reside in halos, the statistics of halos determine
 galaxies on large scale. So the study of halos reconstruction is helpful
 for studying the reconstruction with galaxy surveys, 
 which we will study in the future.
 
%==================================================================
% halo model and properties; biased tracer; what should we be cautious to in reconstruction.
On large scale, the relation between halo and dark matter can be modeled by
a bias factor $b(k)$ and numerous studies have been made about it
\citep{1986ApJ...304...15B,1996MNRAS.282..347M,1999MNRAS.308..119S,2001MNRAS.323....1S,2004MNRAS.355..129S,2011MNRAS.415..383M,2010ApJ...724..878T,2011JCAP...10..031B,2016MNRAS.458.1510S}.
As a biased tracer of dark matter, halos with different mass trace dark matter
in different degrees. 
{\color{blue}So can we get a better result in reconstruction by 
dealing different halos with different ways is an interesting thing.}
In addition, the shot noise, which due to the discrete sampling of halos,
always assumed to the inverse of the number density $1/\bar{n}$, is another
 problem that should be solved. 
 Here, we apply a wiener filter to suppress the shot noise, which
 we will introduce later.\\
%==================================================================

% The relationinvestigate
%between the galaxy and dark matter halos be described by the halo model
%\citep{Halo_model1:2001ApJ,Halo_model2:2000MNRAS,Halo_model3:2000MNRAS}.
%Therefore, understanding tidal reconstruction in view of halo field is
%important for us to extract information from galaxy surveys, which we will
% study in our next work. In this
%paper, we extend 3D tidal reconstruction in view of halos and try to
%reconstruction with a bias-weighting halos field to get a higher correlation
%between reconstructed and original density field.(*)
%==================================================================
%
% In our 
%previous studies \citep{2012:pen,2015:zhu}, the gravitational coupling
%between the long wavelength tidal field and small scale density fluctuations
%can be used to reconstruct the large-scale density field, which provides a new
%method (cosmic tidal reconstruction) to extract the large-scale information from small-scale observables. This provides many independent samples to reduce the sample variance of the measurement of large-scale structure. In addition,
%a developed 3D method obtains even a better result \citep{tides3d}. \\
%%==================================================================

%The basic idea of the cosmic tidal reconstruction is that the local
%anisotropic structure includes the large scale information because of the
%tidal interactions in different scale modes. Each independent measurement 
%of the small scale power spectrum gives some information about the power
%spectrum on large scales. So we can use cheaper local observation to 
%estimate the long wavelength tidal field by cosmic tidal reconstruction,
% which can be used to recover lost 21cm modes and reduce sample 
% variance \citep{2012:pen}. We have tested the reconstruction with simulated 
%dark matter density fields \citep{2015:zhu} and have developed a 
%three-dimensional (3D) method which obtains even better result (in
%preparation). However, the reconstruction with dark matter density field
%can not help us to extend this method to actual data in surveys and more 
%effects should be done.\\

%Galaxy survey is one of the leading methods to measure the clustering
%of dark matter and has been



%  
%The shot noise, which due to the discrete sampling of halos, is another
% problem that should be solved. Generally speaking, we assume the shot
% noise is the inverse of the number density $1/\bar{n}$. 
% So for the dark matter
% density field, the shot noise is too small to ignore. However, the 
% influence of the shot noise limits the result of reconstruction in view of 
% hales. Here, we apply a wiener filter to suppress the shot noise, which
% we will introduce later.\\
 
%Galaxy survey has been one of the leading methods to measure the clustering
%of dark matter. There have been a lot of efforts in developing optimal 
%weighting in galaxy survey \citep{OW1,OW2,OW3,OW4,OW5}. It is difficult to 
%get the mass information in galaxy survey. However we can apply a biased
%weighting, that bias can be given by galaxy luminosity, to improve the 
%signal to noise ratio of the halo fields. In this paper, we are not try to
%find a optimal bias weighting, but the optimal weighting may be needed in
%some case.\\

(*)This paper is organized as follows. In Section \ref{sec:review}, 
we review the
 reconstruction technique. In Section \ref{sec:performance},
  we study the performance of
  reconstruction in halo density fields from $N$-body simulations. 
 In Section \ref{sec:cross_term}, we study how much improvement can 
 get from bias weighted halo fields.\\
 


\section{Review of Tidal Reconstruction}
%\begin{equation}
%a+n S=4
%\label{equ:eq.2}
%\end{equation}
%cite as Eq.(\ref{equ:eq.2})
The basic idea of cosmic tidal reconstruction is presented originally in 
Ref. \citep{2012:pen} and detailed description can be found in 
Ref. \citep{2015:zhu}.
In this section we will briefly review it.\\

%=============================================================================
Not only the small-scale density fluctuation but also the long-wavelength
 tidal field influence the evolution of local small-scale density field. So
 we can use small scale observations to solve the large scale tidal shear 
 and gravitational potential. If we just assume the leading-order coupling
 between the long-wavelength tidal field and small-scale density field, 
  the tidal distortion of the local small-scale power spectra can be 
  written as
\begin{equation}
P(\mathbf{k},\tau)|_{t_{ij}}=P_{1s}(k,\tau)+\hat{k}^{i}\hat{k}^{j}t_{ij}^{(0)}P_{1s}(k,\tau)f(k,\tau),
\label{equ:Pk_s}
\end{equation}
where $\hat{k}$ is the unit vector in the direction of $\mathbf{k}$, 
$P_{1s}$ is the small-scale power spectrum, the long-wavelength 
tidal field is $t_{ij}=\Phi_{L,ij}-\delta_{ij}\nabla^{2}\Phi_{L}/3$ and 
$f(k,\tau)$ are used to describe the coupling of the 
long-wavelength
tidal field and small-scale density field. In this paper, we use superscript
$(0)$ and $\tau$ to denotes the "initial" time and conformal time,
respectively. \\

The filter $f(\mathbf{k},\tau)$ is defined as
\begin{equation}
f(\mathbf{k},\tau)=2\alpha -\beta \frac{d\ln P_{1s}(k,\tau) }{d\ln k},
\label{equ:f}
\end{equation}
where 
\begin{equation}
\begin{split}
&\alpha(\tau)=-D_{\sigma 1}(\tau)+F(\tau),\\
&\beta(\tau)=F(\tau),
\end{split}
\end{equation}
with 
\begin{equation}
D_{\sigma 1}(\tau)=\int^{\tau}_{0}d\tau'\frac{H(\tau)D(\tau')-H{\tau'}D{\tau}}{\dot{H}(\tau')D(\tau')-H(\tau')\dot{D}(\tau')}\frac{T(\tau')D(\tau')}{D(\tau)},
\end{equation}
\begin{equation}
F(\tau)=\int^{\tau}_{0}d\tau''a(\tau'')T(\tau'')\int^{\tau}_{\tau''}d\tau'/a(\tau').
\end{equation}
Here, $D(\tau)$ is the linear growth function, $T(\tau)=D(\tau)/a(\tau)$ is the linear transfer function and $H(\tau)=d\ln a/d\tau$ is the comoving Hubble parameter.\\

From Eq. \eqref{equ:Pk_s}, we express small-scale density fluctuations
with the long-wavelength tidal field $t_{ij}$ and local small-scale
density field $\delta_{1s}$. Then a tidal shear estimator can be constructed
as following.\\

The tidal tensor $t_{ij}$ can be decomposed as
\begin{equation}
t_{ij}=
\begin{pmatrix}
\gamma_{1}-\gamma_{z} & \gamma_{2} & \gamma_{x} \\
\gamma_{2} & -\gamma_{1}-\gamma_{z} & \gamma_{y} \\
\gamma_{x} & \gamma_{y} & 2\gamma_{z}
\end{pmatrix},
\label{tij_c5}
\end{equation}
where
$\gamma_{1}=(\Phi_{L,11}-\Phi_{L,22})/2$,
$\gamma_{2}=\Phi_{L,12}$,
$\gamma_{x}=\Phi_{L,13}$,
$\gamma_{y}=\Phi_{L,23}$,
$\gamma_{z}=(2\Phi_{L,23}-\Phi_{L,11}-\Phi_{L,22})/6$.\\

In 3D reconstruction,
%==================================================================
%\subsection{Tidal Shear Estimators And Density Reconstruction Algorithm}
%The Jacobian matrix can be described the density contrast (Eq. (\ref{equ:Jacobian})) as
% well as the mapping between the source and image planes in gravitational lensing.  %\citep{2008lu}. 
%So we apply CMB lensing techniques to cosmic tidal reconstruction. 
%
%In the long wavelength limit and under the Gaussian assumption, quadratic estimators can be constructed either by using the maximum likelihood method %\citep{2008lu}
%or the inverse variance weighting \citep{2010lu,2012bucher}:
%\begin{equation}
%\hat{\gamma}_1=\frac{1}{Q}\int \frac{d^3k}{(2\pi)^3} \frac{|\delta_g(\mathbf{k})|^2}{L^3}\frac{P(k)}{P^2_{tot}(k)}f(k)(\hat{k}_{1}^{2}-\hat{k}^{2}_{2}),
%\end{equation}
%\begin{equation}
%\hat{\gamma}_2=\frac{1}{Q}\int \frac{d^3k}{(2\pi)^3} \frac{|\delta_g(\mathbf{k})|^2}{L^3}\frac{P(k)}{P^2_{tot}(k)}f(k)(2\hat{k}_{1}\hat{k}_{2}).
%\end{equation}

%If we assume that the long wavelength tidal shear fields vary more slowly than the small scale density field and the tidal shear field is constant in space, the unbiased minimum variance estimates of the spatial varying tidal field in the long wavelength limit is given by
%==================================================================
\begin{equation}
\label{equ:gamma}
\begin{split}
&\hat{\gamma}_{1}=\left[\delta^{w_{1}}_{g}(x)\delta^{w_{1}}_{g}(x)-\delta^{w_{2}}_{g}(x)\delta^{w_{2}}_{g}(x) \right],\\
&\hat{\gamma}_{2}=[2\delta^{w_{1}}_{g}(x)\delta^{w_{2}}_{g}(x)],\\
&\hat{\gamma}_{x}=[2\delta^{w_{1}}_{g}(x)\delta^{w_{3}}_{g}(x)],\\
&\hat{\gamma}_{y}=[2\delta^{w_{2}}_{g}(x)\delta^{w_{3}}_{g}(x)],\\
&\hat{\gamma}_{z}=[2\delta^{w_{3}}_{g}(x)\delta^{w_{3}}_{g}(x)
-\delta^{w_{1}}_{g}(x)\delta^{w_{1}}_{g}(x)
-\delta^{w_{2}}_{g}(x)\delta^{w_{2}}_{g}(x)]/3,
\end{split}
\end{equation}
where $\delta^{w_{i}}_{g}(x)$ is a filtered density fields,and $i$ indicates $\hat{x},\hat{y},\hat{z}$ directions. In Fourier space $w_{i}$ is given by
\begin{equation}
\delta^{w_{i}}_{g}(\mathbf{k})=\delta_{g}(\mathbf{k})w_{i}(\mathbf{k}).
\end{equation}
Here $\delta_{g}$ is the Gaussianized density field and $w_{i}(\mathbf{k})$ is a optimal filter, defined as 
\begin{equation}
w_{i}(\mathbf{k})=\ii \hat{k}_{i} \left[\frac{P(k)f(k)}{QP^{2}_{tot}(k)}\right]^{1/2},
\label{equ:w}
\end{equation}
with 
\begin{equation}
Q=\int \frac{2k^{2}dk}{15\pi^{2}}\frac{P^{2}(k)}{P^{2}_{tot}(k)}f^{2}(k)
\label{equ:Q}
\end{equation}
and
\begin{equation}
P_{tot}(k)=P(k)+P_{N}(k)
 \end{equation}
 is the observed power spectrum which includes both signal and noise. The filter $f(k)$ is from Eq. \eqref{equ:f}.\\
 
 Then the reconstructed 3D convergence field is
 \begin{equation}
 \label{equ:kappa3D}
 \begin{split}
 \kappa_{3\mathrm{D}}(\mathbf{k})=
 \frac{1}{k^{2}}
 [
&(k_{1}^{2}-k_{2}^{2})\gamma_{1}(\mathbf{k})
 +2k_{1}k_{2}\gamma_{2}(\mathbf{k}) 
 +2k_{1}k_{3}\gamma_{x}(\mathbf{k})\\
&+2k_{2}k_{3}\gamma_{y}(\mathbf{k})
 +(2k_{3}^2-k_1^2-k_2^2)\gamma_{z}(\mathbf{k})
 ].
 \end{split}
 \end{equation}
%========wiener of kappa================================================
To correct the bias and reduce the noise in reconstructed field,
 we write the reconstructed clean field $\hat{\kappa}$ as 
\begin{equation}
\label{eq: kappa}
\hat{\kappa}(k_{\perp},k_{\parallel})=\frac{\kappa_{3\mathrm{D}}(k_{\perp},k_{\parallel})}{b(k_{\perp},k_{\parallel})} W(k_{\perp},k_{\parallel}),
\end{equation}
where bias factor 
\begin{equation}
b(k_{\perp},k_{\parallel})=\frac{P_{\kappa_{3\mathrm{D}}\delta}(k_{\perp},k_{\parallel})}{P_{\delta}(k_{\perp},k_{\parallel})}
\label{equ:bias}
\end{equation}
 and Wiener filter 
\begin{equation}
W(k_{\perp},k_{\parallel})=\frac{P_{\delta}(k_{\perp},k_{\parallel})}{P_{\kappa_{3\mathrm{D}}}(k_{\perp},k_{\parallel})/b^{2}(k_{\perp},k_{\parallel})}.
\end{equation}
Here, the noise power spectrum is 
\begin{equation}
P_{n}(k_{\perp},k_{\parallel})=P_{\kappa 3\mathrm{D}}(k_{\perp},k_{\parallel})-b^2(k_{\perp},k_{\parallel})P_{\delta}(k_{\perp},k_{\parallel}).
\end{equation}


%Add here the basic
%physical idea about reconstruction large scale density field from small scale
%tidal deformations, the tidal shear estimators, and the reconstruction 
%algorithm. 


%Discuss why we try to use halos in this paper: we can only observe
%galaxies instead of dark matter density fields, halos are unperfect tracers
%of the underlying dark matter density field. also add some figures containing
%both the distribution of the dark matter density field and the halos here.
%
%Remainder: most content in this section can be directly copy+paste from my
%original paper. Don't be silly trying to type all those equations again !!!
%=======================================

\section{Performance of reconstruction}
we have discussed the tidal reconstruction with local dark matter density field \citep{2015:zhu}. In practice, however, we can only observe galaxies instead of dark matter density fields. And the distribution of halo is an unperfect tracer of the underlying dark matter density. Therefore, the reconstruction in view of halo is even more relevant to the observation.

In this section we perform reconstruction in halos from $N$-body simualtions,
following the same process in the previous paper. Use cross correlation 
coefficient to quantify the reconstruction results and also discuss the 
anisotropic noise.

%=======================================
\subsection{Simulation}
We run N-body simulations using the $\mathrm{CUBEP^3M}$ code with $2048^3$ dark matter particles in a box of side length $L=1.2\ \mr{Gpc}/h$. We have adopted the following set of cosmological parameter values: $\Omega_{b}=0.049$, $\Omega_{c}=0.259$, $h=0.678$, $A_s=2.139\times 10^{-9}$ and $n_s=0.968$. Six runs with independent initial conditions were performed to provide better statistics. In the following calculations we use outputs at $z=0$. 
\\copyed from zhu,maybe add something by YuYu.
\input{./split/halo.tex}
%
%\subsection{Reconstruction}
%The performance of reconstruction we follow our previous work \citep{2012:pen,2015:zhu}. \\
%(i). Convolving the halo density field with a filter $W$ defined as 
%\begin{equation}
%W(\mb{k})=\frac{P_{h}(\mb{k})}{P_{h}(\mb{k})+1/\bar{n}},
%\label{eq: nbar_filter}
%\end{equation}
%where $P_{h}$ and $\bar{n}$ is the power spectrum and the number density of the halo filed. Then we obtain the filtered halo density field. \\
%%$\tilde{\delta}$.\\% Because \\
%(ii). Using a Gaussian window function as a smooth kernel to smooth $\delta_{h}$,
%\begin{equation}
%\bar{\delta}(\mb{x})=\int d^{3}x'\mb{S}(\mb{x}-\mb{x}')\delta	(\mb{x}'),
%\label{eq: smooth kernal}
%\end{equation}
%where $\mb{S}(r)=\e^{-r^{2}/2R^{2}}$. The smoothing scale we select as $R=1.25Mpc/h$, which is the same as in the reconstruction of dark matter \cite{2015:zhu}. Then we Gaussianize the density field like %$G(1+\bar{\delta}_{h})$ like. %\cite{1992:weinberg_Gau}.
%\\(iii). We convolve the smoothed and Gaussianized density field with $w_{i}(\mb{x})$ (Eq. (\ref{eq:w})). Then we obtain the three dimensional tidal shear $\gamma_{1}(x)$ and $\gamma_{2}(x)$. Calculating $\gamma_{1}$ and $\gamma_{2}$ as Eq.(\ref{eq: k_3d}), we obtain 3D convergence field $\kappa_{3D}(\mb{k})$. 
%\\(iv). To calculate $b(k_{\perp},k_{\parallel})$ and $W(k_{\perp},k_{\parallel})$ to correct the estimated $\kappa_{3D}$, we estimate the power spectrum $P_{\kappa_{3D}}$ and $P_{\kappa_{3D}\delta}$ use these six simulations. Then we obtain the reconstructed clean $\hat{\kappa(\mb{k})}$ from Eq. (\ref{eq: kappa}).
%
%%==========result=======================
%\subsection{Results}
%%1. show the slice of kappa field
%%\input{./eps/slice.tex}
%
%%2. show the PS and CC
%\input{./eps/PS.tex}
%\input{./eps/CC.tex}
%In Fig. \ref{fig:PS}, we show both the auto power spectrum of the simulated original 
%dark matter density field $\delta$ and corresponding halo density field 
%$\delta_{h}$. Also, the result of reconstructed field $\kappa$ and the cross power 
%spectrum between $\kappa$ and $\delta$ is demonstrated. We move the data points in 
%the same $k$-bin slightly to show lines clearly. In this paper, we use haloes to 
%trace the underlying dark matter density field of the large scale structure. 
%However, it is a biased tracer \citep{2016:halo_bias} which we can find from Fig. 
%\ref{fig:PS}. From Eq. (\ref{eq: kappa}), the bias can be included in 
%$b(k_{\perp},k_{\parallel})$ so we needn't extra processing for this effect in our 
%reconstruction. We find there is a large error in small $k$ which is because of the 
%finite modes in large scale. And fortunately, we get a good correlation coefficient 
%$r > 0.7$ in almost all $k$ up to $0.05Mpc/h$, and a correlation coefficient more 
%than $0.5$ even in $k<0.1Mpc/h$. We have discussed in the previous paper that the 
%reconstruction does not work well at $k>0.1Mpc/h$ \citep{2015:zhu}. Here we define 
%correlation coefficient as $r\equiv P_{\delta\kappa}/ \sqrt{P_{\delta}P_{\kappa}}$.
%
%In Fig. \ref{fig:cc}, we show the correlation coefficient $r$ in different smoothing scale. The error bars are estimated by the bootstrap resampling method. We have discussed the effect of smoothing and Gaussianization in previous work \citep{2015:zhu}. Here we obtain the same conclusion and use the optimal smoothing scale $1.25\ \mr{Mpc}/h$ in the following discuss.
%
%%defferent number density of halo
%\input{./eps/number_density.tex}
%\input{./eps/mass.tex}
%Until now, the number densities of halos fields we used are $4.8\times 10^{-3}\ h^{3}/\mr{Mpc}^{3}$. However, different from dark matter density field, haloes field is sensitive to shot noise. So the effect of the shot noise must be considered in reconstruction, see Fig. \ref{fig:ND}. Obviously, higher number density corresponds to better correlation coefficient. Fortunately, we can obtain the data of galaxies with even higher number density than $4.8\times 10^{-3}\ h^{3}/\mr{Mpc}^{3}$ from current or upcoming surveys. add cite. Therefore, it is worth looking forward to the reconstruction by the data of galaxy survey.
%
%In Fig. \ref{fig:mass}, we resample the haloes field according to halo mass and reconstruct in the same way. All subsamples have the same number density $\bar{n}=1.2 \times 10^{-3}\ h^{3}\mr{Mpc}^{-3}$. The result shows that the high-mass haloes subsample has better  reconstruction performance.
%%add something.
%
%\subsection{The Anisotropic Noise}
%%3. show the 2d correlation coefficient and noise and bias.
%The following content we default the reconstruction is optimal, with $1.25\ \mr{Mpc}/h$ smoothing scale, $4.8 \times 10^{-3}\ h^3\mr{Mpc}^{3}$ number density and Gaussianize.
%\input{./eps/Pn.tex}
%\input{./eps/2D_CC.tex}
%The error in the reconstructed field $\kappa_{3D}(\mb{k})$ is anisotropic and
% $\sigma^{2}_{3D}\propto (k^{2}/k^{2}_{\perp})^{2}$ \citep{2015:zhu}. In Fig.  \ref{fig:2d_Pn},  we show the 2D noise power spectrum. we find there are large noise in modes 
% with large $k_{\parallel}$ and small $k_{\perp}$, which is consistent with our prediction.
%
%In Fig. \ref{fig:2D_cc}, we show the 2D correlation coefficient which defined as 
%\begin{equation}
%r_{\kappa \delta}(k_{\parallel},k_{\perp})=P_{\kappa \delta}(k_{\parallel},k_{\perp})\sqrt{P_{\delta}(k_{\parallel},k_{\perp})P_{\kappa}(k_{\parallel},k_{\perp})}.
%\end{equation}
%We find the correlation coefficient decreases quickly with the increase of $k_{\parallel}$. As previously mentioned, we only use the 2-D tidal shear fields in estimator. Therefore, the reconstruction is more sensitive to noise and has less correlation coefficient in the large $k_{\parallel}$ mode. 
%
%%=======================================
%%4. show 2d bias
%\subsection{bias factor}
%\input{./eps/bias.tex}
% If we look back the entire process of reconstruction, the possible bias may comes from the following aspects. (i) We use halo field as a biased trace of dark matter density field. (ii) We thought the integral of $Q$ in \eqref{equ:Q} would diverge at large $k$ if there were no noise in the power spectrum $P(k)$ in our previous paper \citep{2015:zhu}. But in simulation, we can calculate it actually. (iii) At last, the tidal shear estimators are biased and it is what we want to quantified. 
%
%In Fig. \ref{fig:2d_bias}, we show the 2-D bias factor $b(k_{\parallel},k_{\perp})$ which contains all of the three terms.
%\section{Dependence on redshift space distortion}
%%First discuss why we want to study the RSD halos, like what we observe is
%%the RSD halos, etc....
%%
%%In this section we perform reconstruction in halos which positions has been
%%shifted according to the $z$ component of pecular velocities. Use the some
%%quantites to quantify the anisotropic features in reconstruction due RSD, 
%%like the bias factor.
%%=======================================
%In reality, our observation is the density field in redshift space. Until now, everything we have done is in real space, ignoring the redshift space distortions (RSD) that arise from peculiar velocities. In this section, we will perform reconstruction in haloes which positions has been shifted according to the $z$ component of pecular velocities. After this, we discuss the effect of RSD for reconstruction and attempt to remove the RSD signal during the reconstruction like %\citep{rsd:saito,rsd:Anderson}.
%\subsection{Effect of RSD}
%\input{./eps/PS_rsd.tex}
%\input{./eps/CC_rsd.tex}
%
%
%We add RSD effect in simulation and reconstruct large scale density field. %add something, such as how to add RSD effect.
%In Fig. \ref{fig:PS_rsd}, we show the auto and cross power spectra of the original
%dark matter fields and reconstructed density fields from RSD halo field. Compared with Fig. \ref{fig:PS}, the auto power spectra of halo fields $P_{h}$ become higher, which is because the RSD effect improves the fluctuation of halo density field.
%
%In Fig. \ref{fig:cc_rsd}, we plot cross correlation coefficients of reconstructed tidal field from RSD halo field and dark matter density field. green, blue and red denote with, without and removing RSD effect, respectively. And the process of removing the effect of RSD we will describe later. We find difference between the correlation coefficient of reconstruction by halo field with and without RSD effect is about $0.05$. After we remove the effect of RSD, the correlation coefficient is almost back the same order with no RSD halo fields.
%
%\begin{figure}[tbp]
%\begin{center}
%\includegraphics[width=0.48\textwidth]{./eps/rsd_2d_bias.eps}
%\end{center}
%\vspace{-0.7cm}
%\caption{RSD: 2D bias factor $b(k_{\parallel},k_{\perp})$.}
%%\label{fig:2d_bias}
%\end{figure}
%
%\begin{figure}[tbp]
%\begin{center}
%\includegraphics[width=0.48\textwidth]{./eps/rsd_2d_Pn_withbias.eps}
%\end{center}
%\vspace{-0.7cm}
%\caption{RSD: The anisotropic noise power spectrum $P_{n}(k_{\parallel},k_{\perp})$.}
%%\label{fig:2d_bias}
%\end{figure}
%
%
%\subsection{Removing the RSD effect}as previously discussed
%The RSD effect causes a bad influence to reconstruction, as we previously discussed. Here, we follow the method of isotropic BAO reconstruction
%% \citep{rsd:saito,rsd:Anderson} 
% to remove the effect of RSD.
% 
%If we write the halo density field as $\delta(x)$, then we can estimate the displacement field $\tilde{s(k)}$ in Fourier space as:
%\begin{equation}
%\tilde{s}(k) \approx -\frac{i\mathbf{k}}{k^2}\tilde{\delta}_{s}(k),
%\end{equation}
%Where $\tilde{\delta}_{s}(k)$ is the smoothed density field use Eq. (\ref{eq: smooth kernal}). Different from BAO reconstruction, the smoothing scale here we just use $1.25\ \mr{Mpc}/h$.
%
%On linear scales, RSD effect enhance the halo density field as:
%\begin{equation}
%\tilde{\delta^{s}_{h}}=b(1+\beta\mu^{2})\tilde{\delta^{r}_{m}},
%\end{equation}
%where b is the linear halo bias, $\tilde{\delta^{r}_{m}}$ is the real space dark matter density field, $\tilde{\delta^{s}_{h}}$ is the redshift space halo density field, $\beta=f/b$ and $\mu$ is the cosine of the angle between the line of sight and save vector.
%
%So, the real space displacement field $\tilde{s}^{r}(k)$ can be estimated as:
%\begin{equation}
%\tilde{s}^{r}(\mb{k})=-\frac{i\mb{k}}{k^2}\frac{\tilde{\delta}_{s}(\mb{k})}{b(1+\beta \mu^{2})}.
%\end{equation}


\section{Discussion}
Discuss these reconstruction results and its future applications....
%=======================================


\section{Acknowledgement}
We acknowledge the support of the Chinese MoST 863 program under Grant 
No. 2012AA121701, the CAS Science Strategic Priority Research Program 
XDB09000000, the NSFC under Grant No. 11373030, Tsinghua University, 
CHEP at Peking University, and NSERC.
%=======================================

\input{./eps/slice.tex}
\bibliographystyle{apsrev}
%\bibliographystyle{apj}
%\bibliographystyle{mnras}
\bibliography{tide}

\end{document}
