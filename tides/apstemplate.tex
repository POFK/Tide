%% ****** Start of file apstemplate.tex ****** %
%%
%%
%%   This file is part of the APS files in the REVTeX 4 distribution.
%%   Version 4.1r of REVTeX, August 2010
%%
%%
%%   Copyright (c) 2001, 2009, 2010 The American Physical Society.
%%
%%   See the REVTeX 4 README file for restrictions and more information.
%%
%
% This is a template for producing manuscripts for use with REVTEX 4.0
% Copy this file to another name and then work on that file.
% That way, you always have this original template file to use.
%
% Group addresses by affiliation; use superscriptaddress for long
% author lists, or if there are many overlapping affiliations.
% For Phys. Rev. appearance, change preprint to twocolumn.
% Choose pra, prb, prc, prd, pre, prl, prstab, prstper, or rmp for journal
%  Add 'draft' option to mark overfull boxes with black boxes
%  Add 'showpacs' option to make PACS codes appear
%  Add 'showkeys' option to make keywords appear
%=============================================================================
%\documentclass[aps,prd,preprint,groupedaddress]{revtex4-1}
%\documentclass[aps,prl,preprint,superscriptaddress]{revtex4-1}
%\documentclass[aps,prd,reprint,groupedaddress]{revtex4-1}
%\documentclass[aps,prd,twocolumn,groupedaddress]{revtex4-1}
\documentclass[aps,prd,twocolumn,showpacs,superscriptaddress,groupedaddress,
nofootinbib,floatfix]{revtex4}  % for review and submission
%\documentclass[aps,prl,twocolumn,showpacs,superscriptaddress,groupedaddress,
%nofootinbib,floatfix]{revtex4}
% You should use BibTeX and apsrev.bst for references
% Choosing a journal automatically selects the correct APS
% BibTeX style file (bst file), so only uncomment the line
% below if necessary.
%\bibliographystyle{apsrev4-1}
%=============================================================================
%\documentclass[aps,prd,twocolumn,showpacs,superscriptaddress,groupedaddress,nofootinbib]{revtex4}  % for review and submission
%\documentclass[aps,preprint,showpacs,superscriptaddress,groupedaddress]{revtex4}  % for double-spaced preprint
\usepackage{graphicx}  % needed for figures
\usepackage{dcolumn}   % needed for some tables
\usepackage{bm}        % for math
\usepackage{amsmath,amssymb}   % for math
\usepackage{color}
\usepackage{multirow}
%=============================================================================
%\usepackage{graphicx}  % needed for figures
%%\usepackage{dcolumn}   % needed for some tables
%\usepackage{bm}        % for math
%\usepackage{amsmath,amssymb}   % for math
%\usepackage{color}
%% avoids incorrect hyphenation, added Nov/08 by SSR
%\hyphenation{ALPGEN}
%\hyphenation{EVTGEN}
%\hyphenation{PYTHIA}
\newcommand{\mnras}{Mon. Not. R. Astron. Soc.}
\newcommand{\mv}{\mathbf}
\newcommand{\ii}{\mathrm{i}}
%=============================================================================
\begin{document}

% Use the \preprint command to place your local institutional report
% number in the upper righthand corner of the title page in preprint mode.
% Multiple \preprint commands are allowed.
% Use the 'preprintnumbers' class option to override journal defaults
% to display numbers if necessary
%\preprint{}

%Title of paper
\title{Cosmic tides in view of halos}

% repeat the \author .. \affiliation  etc. as needed
% \email, \thanks, \homepage, \altaffiliation all apply to the current
% author. Explanatory text should go in the []'s, actual e-mail
% address or url should go in the {}'s for \email and \homepage.
% Please use the appropriate macro foreach each type of information

% \affiliation command applies to all authors since the last
% \affiliation command. The \affiliation command should follow the
% other information
% \affiliation can be followed by \email, \homepage, \thanks as well.
%\author{}
%\email[]{Your e-mail address}
%\homepage[]{Your web page}
%\thanks{}
%\altaffiliation{}
%\affiliation{}
\author{Tian-Xiang Mao}
\affiliation{Key Laboratory for Computational Astrophysics,
National Astronomical Observatories, Chinese Academy of Sciences,
20A Datun Road, Beijing 100012, China}
\affiliation{University of Chinese Academy of Sciences, Beijing 100049, China}

\author{Hong-Ming Zhu}
\affiliation{Key Laboratory for Computational Astrophysics,
National Astronomical Observatories, Chinese Academy of Sciences,
20A Datun Road, Beijing 100012, China}
\affiliation{University of Chinese Academy of Sciences, Beijing 100049, China}

\author{Ue-Li Pen}
\affiliation{Canadian Institute for Theoretical Astrophysics, University of Toronto, 60 St. George Street, Toronto, Ontario M5S 3H8, Canada}
\affiliation{Dunlap Institute for Astronomy and Astrophysics, University of Toronto, 50 St. George Street, Toronto, Ontario M5S 3H8, Canada}
\affiliation{Canadian Institute for Advanced Research, CIFAR Program in Gravitation and Cosmology, Toronto, Ontario M5G 1Z8, Canada}
\affiliation{Perimeter Institute for Theoretical Physics, 31 Caroline Street North, Waterloo, Ontario, N2L 2Y5, Canada}

\author{Yu Yu}
\affiliation{Key laboratory for research in galaxies and cosmology,
Shanghai Astronomical Observatory, Chinese Academy of Sciences,
80 Nandan Road, Shanghai 200030, China}

\author{Xuelei Chen}
\affiliation{Key Laboratory for Computational Astrophysics,
National Astronomical Observatories, Chinese Academy of Sciences,
20A Datun Road, Beijing 100012, China}
\affiliation{University of Chinese Academy of Sciences, Beijing 100049, China}
\affiliation{Center of High Energy Physics, Peking University, Beijing 100871, China}

\author{Jie Wang}
\affiliation{Key Laboratory for Computational Astrophysics,
National Astronomical Observatories, Chinese Academy of Sciences,
20A Datun Road, Beijing 100012, China}
%\affiliation{University of Chinese Academy of Sciences, Beijing 100049, China}

%Collaboration name if desired (requires use of superscriptaddress
%option in \documentclass). \noaffiliation is required (may also be
%used with the \author command).
%\collaboration can be followed by \email, \homepage, \thanks as well.
%\collaboration{}
%\noaffiliation

\date{\today}
%=============================================================================
\begin{abstract}
% insert abstract here
\end{abstract}

% insert suggested PACS numbers in braces on next line
%\pacs{}
% insert suggested keywords - APS authors don't need to do this
%\keywords{}

%\maketitle must follow title, authors, abstract, \pacs, and \keywords
\maketitle

% body of paper here - Use proper section commands
% References should be done using the \cite, \ref, and \label commands
%\section{}
% Put \label in argument of \section for cross-referencing
%\section{\label{}}
%\subsection{}
%\subsubsection{}

% If in two-column mode, this environment will change to single-column
% format so that long equations can be displayed. Use
% sparingly.
%\begin{widetext}
% put long equation here
%\end{widetext}

% figures should be put into the text as floats.
% Use the graphics or graphicx packages (distributed with LaTeX2e)
% and the \includegraphics macro defined in those packages.
% See the LaTeX Graphics Companion by Michel Goosens, Sebastian Rahtz,
% and Frank Mittelbach for instance.
%
% Here is an example of the general form of a figure:
% Fill in the caption in the braces of the \caption{} command. Put the label
% that you will use with \ref{} command in the braces of the \label{} command.
% Use the figure* environment if the figure should span across the
% entire page. There is no need to do explicit centering.

% \begin{figure}
% \includegraphics{}%
% \caption{\label{}}
% \end{figure}

% Surround figure environment with turnpage environment for landscape
% figure
% \begin{turnpage}
% \begin{figure}
% \includegraphics{}%
% \caption{\label{}}
% \end{figure}
% \end{turnpage}

% tables should appear as floats within the text
%
% Here is an example of the general form of a table:
% Fill in the caption in the braces of the \caption{} command. Put the label
% that you will use with \ref{} command in the braces of the \label{} command.
% Insert the column specifiers (l, r, c, d, etc.) in the empty braces of the
% \begin{tabular}{} command.
% The ruledtabular enviroment adds doubled rules to table and sets a
% reasonable default table settings.
% Use the table* environment to get a full-width table in two-column
% Add \usepackage{longtable} and the longtable (or longtable*}
% environment for nicely formatted long tables. Or use the the [H]
% placement option to break a long table (with less control than 
% in longtable).
% \begin{table}%[H] add [H] placement to break table across pages
% \caption{\label{}}
% \begin{ruledtabular}
% \begin{tabular}{}
% Lines of table here ending with \\
% \end{tabular}
% \end{ruledtabular}
% \end{table}

% Surround table environment with turnpage environment for landscape
% table
% \begin{turnpage}
% \begin{table}
% \caption{\label{}}
% \begin{ruledtabular}
% \begin{tabular}{}
% \end{tabular}
% \end{ruledtabular}
% \end{table}
% \end{turnpage}

% Specify following sections are appendices. Use \appendix* if there
% only one appendix.
%\appendix
%\section{}
%=============================================================================
% body of paper here
\section{\label{sec:introduction}Introduction}
The large-scale structure contains a wealth of information about our Universe.
By measuring large-scale structure we can attempt to
find the answers to such fundamental questions as what the initial
conditions of the Universe are and what its future will be.
{\color{red}There have been a lot of efforts in measuring the large-scale
structure of our Universe (e.g. SDSS,).} 
However, the mode coupling because of nonlinear evolution 
{\color{blue} (here add citation)} and many other observational
problems {\color{red}(such as RSD)} make it difficult to extract cosmological information. 
Additionally, because of the limited volume of surveys, uncertainties 
due to sample variance is also difficult to overcome.
%==================================================================

Dark matter halos, which host observable galaxies and galaxy clusters, 
are the fundamental nonlinear units of cosmic structure. On large scales,
 dark matter halos can be considered as biased tracers 
of the underlying dark matter distribution \citep{1984ApJ...284L...9K}.
Not only the local matter distribution but also
the large-scale environment can affect the shapes, the abundance, 
the accretion, the bias, and the power spectrum of halos
\citep{2016ApJ...825...49C,2012JCAP...03..004S,1996MNRAS.282..347M,2015ApJ...807...37S,2015JCAP...09..028C,2014PhRvD..90j3530L,2014JCAP...05..048C,2013PhRvD..87l3504T}.
{\color{blue}
So the coupling between the large-scale environment and small-scale 
{\color{red}density fluctuation} may be used to extract more information about 
the large-scale structure, which is helpful to suppress the sample variance
of large-scale modes.}
In recent studies, this effect is used to get the 
information of very long-wavelength modes
outside the survey volume by considering the large-scale environment
contribute to the mean density fluctuation in the survey
(or "separate universe")
\citep{2016JCAP...09..007B,2016arXiv160901701C,2016PhRvD..93f3507L,2016JCAP...02..018L, 2015JCAP...08..042W,2003ApJ...585...34M}.
Nevertheless, the isotropic distortions on small scales may derived from
not only the large-scale environment but also some other processes.
In our previous stydies \citep{2012arXiv1202.5804P,2016PhRvD..93j3504Z},
the long-wavelength tidal field is reconstructed by quantifying the local
anisotropy of small-scale density statistics and even a better result
obtained by a developed 3-D method \citep{tides3d:Hu}.
%If we only consider the position of halos, the large-scale density field
%is also can be reconstructed by halos field.
%==================================================================

In this paper, we investigate the cosmic tidal reconstruction in view
of halos.
Different with dark matter density field, reconstruction with halos field has
many advantages. In the dark matter case, we need to take a logarithmic 
transform or map the density fluctuations into a Gaussian distribution to
Gaussianize the smoothed density field or suppress the high weights 
of the high density regions because of quadratic estimator
\citep{2012arXiv1202.5804P,2016PhRvD..93j3504Z,tides3d:Hu}.
However, for halo counts, there is not such problems. Without Gaussianization,
we can quantify the bias of reconstruction easier and get a simpler and 
clearer physical picture, which we will discuss in \ref{sec:performance}. 
Additionally, since most observed tracers of large-scale structure, 
such as galaxies, reside in halos, the statistics of halos determine
 galaxies on large scale. So the study of halos reconstruction is helpful
 for studying the reconstruction with galaxy surveys, 
 which we will study in the future.
 
%==================================================================
% halo model and properties; biased tracer; what should we be cautious to in reconstruction.
On large scale, the relation between halo and dark matter can be modeled by
a bias factor $b(k)$ and numerous studies have been made about it
\citep{1986ApJ...304...15B,1996MNRAS.282..347M,1999MNRAS.308..119S,2001MNRAS.323....1S,2004MNRAS.355..129S,2011MNRAS.415..383M,2010ApJ...724..878T,2011JCAP...10..031B,2016MNRAS.458.1510S}.
As a biased tracer of dark matter, halos with different mass trace dark matter
in different degrees. 
{\color{blue}So can we get a better result in reconstruction by 
dealing different halos with different ways is an interesting thing.}
In addition, the shot noise, which due to the discrete sampling of halos,
always assumed to the inverse of the number density $1/\bar{n}$, is another
 problem that should be solved. 
 Here, we apply a wiener filter to suppress the shot noise, which
 we will introduce later.\\
%==================================================================

% The relationinvestigate
%between the galaxy and dark matter halos be described by the halo model
%\citep{Halo_model1:2001ApJ,Halo_model2:2000MNRAS,Halo_model3:2000MNRAS}.
%Therefore, understanding tidal reconstruction in view of halo field is
%important for us to extract information from galaxy surveys, which we will
% study in our next work. In this
%paper, we extend 3D tidal reconstruction in view of halos and try to
%reconstruction with a bias-weighting halos field to get a higher correlation
%between reconstructed and original density field.(*)
%==================================================================
%
% In our 
%previous studies \citep{2012:pen,2015:zhu}, the gravitational coupling
%between the long wavelength tidal field and small scale density fluctuations
%can be used to reconstruct the large-scale density field, which provides a new
%method (cosmic tidal reconstruction) to extract the large-scale information from small-scale observables. This provides many independent samples to reduce the sample variance of the measurement of large-scale structure. In addition,
%a developed 3D method obtains even a better result \citep{tides3d}. \\
%%==================================================================

%The basic idea of the cosmic tidal reconstruction is that the local
%anisotropic structure includes the large scale information because of the
%tidal interactions in different scale modes. Each independent measurement 
%of the small scale power spectrum gives some information about the power
%spectrum on large scales. So we can use cheaper local observation to 
%estimate the long wavelength tidal field by cosmic tidal reconstruction,
% which can be used to recover lost 21cm modes and reduce sample 
% variance \citep{2012:pen}. We have tested the reconstruction with simulated 
%dark matter density fields \citep{2015:zhu} and have developed a 
%three-dimensional (3D) method which obtains even better result (in
%preparation). However, the reconstruction with dark matter density field
%can not help us to extend this method to actual data in surveys and more 
%effects should be done.\\

%Galaxy survey is one of the leading methods to measure the clustering
%of dark matter and has been



%  
%The shot noise, which due to the discrete sampling of halos, is another
% problem that should be solved. Generally speaking, we assume the shot
% noise is the inverse of the number density $1/\bar{n}$. 
% So for the dark matter
% density field, the shot noise is too small to ignore. However, the 
% influence of the shot noise limits the result of reconstruction in view of 
% hales. Here, we apply a wiener filter to suppress the shot noise, which
% we will introduce later.\\
 
%Galaxy survey has been one of the leading methods to measure the clustering
%of dark matter. There have been a lot of efforts in developing optimal 
%weighting in galaxy survey \citep{OW1,OW2,OW3,OW4,OW5}. It is difficult to 
%get the mass information in galaxy survey. However we can apply a biased
%weighting, that bias can be given by galaxy luminosity, to improve the 
%signal to noise ratio of the halo fields. In this paper, we are not try to
%find a optimal bias weighting, but the optimal weighting may be needed in
%some case.\\

(*)This paper is organized as follows. In Section \ref{sec:review}, 
we review the
 reconstruction technique. In Section \ref{sec:performance},
  we study the performance of
  reconstruction in halo density fields from $N$-body simulations. 
 In Section \ref{sec:cross_term}, we study how much improvement can 
 get from bias weighted halo fields.\\
 


\section{\label{sec:review}Review of Tidal Reconstruction}
%\begin{equation}
%a+n S=4
%\label{equ:eq.2}
%\end{equation}
%cite as Eq.(\ref{equ:eq.2})
The basic idea of cosmic tidal reconstruction is presented originally in 
Ref. \citep{2012:pen} and detailed description can be found in 
Ref. \citep{2015:zhu}.
In this section we will briefly review it.\\

%=============================================================================
Not only the small-scale density fluctuation but also the long-wavelength
 tidal field influence the evolution of local small-scale density field. So
 we can use small scale observations to solve the large scale tidal shear 
 and gravitational potential. If we just assume the leading-order coupling
 between the long-wavelength tidal field and small-scale density field, 
  the tidal distortion of the local small-scale power spectra can be 
  written as
\begin{equation}
P(\mathbf{k},\tau)|_{t_{ij}}=P_{1s}(k,\tau)+\hat{k}^{i}\hat{k}^{j}t_{ij}^{(0)}P_{1s}(k,\tau)f(k,\tau),
\label{equ:Pk_s}
\end{equation}
where $\hat{k}$ is the unit vector in the direction of $\mathbf{k}$, 
$P_{1s}$ is the small-scale power spectrum, the long-wavelength 
tidal field is $t_{ij}=\Phi_{L,ij}-\delta_{ij}\nabla^{2}\Phi_{L}/3$ and 
$f(k,\tau)$ are used to describe the coupling of the 
long-wavelength
tidal field and small-scale density field. In this paper, we use superscript
$(0)$ and $\tau$ to denotes the "initial" time and conformal time,
respectively. \\

The filter $f(\mathbf{k},\tau)$ is defined as
\begin{equation}
f(\mathbf{k},\tau)=2\alpha -\beta \frac{d\ln P_{1s}(k,\tau) }{d\ln k},
\label{equ:f}
\end{equation}
where 
\begin{equation}
\begin{split}
&\alpha(\tau)=-D_{\sigma 1}(\tau)+F(\tau),\\
&\beta(\tau)=F(\tau),
\end{split}
\end{equation}
with 
\begin{equation}
D_{\sigma 1}(\tau)=\int^{\tau}_{0}d\tau'\frac{H(\tau)D(\tau')-H{\tau'}D{\tau}}{\dot{H}(\tau')D(\tau')-H(\tau')\dot{D}(\tau')}\frac{T(\tau')D(\tau')}{D(\tau)},
\end{equation}
\begin{equation}
F(\tau)=\int^{\tau}_{0}d\tau''a(\tau'')T(\tau'')\int^{\tau}_{\tau''}d\tau'/a(\tau').
\end{equation}
Here, $D(\tau)$ is the linear growth function, $T(\tau)=D(\tau)/a(\tau)$ is the linear transfer function and $H(\tau)=d\ln a/d\tau$ is the comoving Hubble parameter.\\

From Eq. \eqref{equ:Pk_s}, we express small-scale density fluctuations
with the long-wavelength tidal field $t_{ij}$ and local small-scale
density field $\delta_{1s}$. Then a tidal shear estimator can be constructed
as following.\\

The tidal tensor $t_{ij}$ can be decomposed as
\begin{equation}
t_{ij}=
\begin{pmatrix}
\gamma_{1}-\gamma_{z} & \gamma_{2} & \gamma_{x} \\
\gamma_{2} & -\gamma_{1}-\gamma_{z} & \gamma_{y} \\
\gamma_{x} & \gamma_{y} & 2\gamma_{z}
\end{pmatrix},
\label{tij_c5}
\end{equation}
where
$\gamma_{1}=(\Phi_{L,11}-\Phi_{L,22})/2$,
$\gamma_{2}=\Phi_{L,12}$,
$\gamma_{x}=\Phi_{L,13}$,
$\gamma_{y}=\Phi_{L,23}$,
$\gamma_{z}=(2\Phi_{L,23}-\Phi_{L,11}-\Phi_{L,22})/6$.\\

In 3D reconstruction,
%==================================================================
%\subsection{Tidal Shear Estimators And Density Reconstruction Algorithm}
%The Jacobian matrix can be described the density contrast (Eq. (\ref{equ:Jacobian})) as
% well as the mapping between the source and image planes in gravitational lensing.  %\citep{2008lu}. 
%So we apply CMB lensing techniques to cosmic tidal reconstruction. 
%
%In the long wavelength limit and under the Gaussian assumption, quadratic estimators can be constructed either by using the maximum likelihood method %\citep{2008lu}
%or the inverse variance weighting \citep{2010lu,2012bucher}:
%\begin{equation}
%\hat{\gamma}_1=\frac{1}{Q}\int \frac{d^3k}{(2\pi)^3} \frac{|\delta_g(\mathbf{k})|^2}{L^3}\frac{P(k)}{P^2_{tot}(k)}f(k)(\hat{k}_{1}^{2}-\hat{k}^{2}_{2}),
%\end{equation}
%\begin{equation}
%\hat{\gamma}_2=\frac{1}{Q}\int \frac{d^3k}{(2\pi)^3} \frac{|\delta_g(\mathbf{k})|^2}{L^3}\frac{P(k)}{P^2_{tot}(k)}f(k)(2\hat{k}_{1}\hat{k}_{2}).
%\end{equation}

%If we assume that the long wavelength tidal shear fields vary more slowly than the small scale density field and the tidal shear field is constant in space, the unbiased minimum variance estimates of the spatial varying tidal field in the long wavelength limit is given by
%==================================================================
\begin{equation}
\label{equ:gamma}
\begin{split}
&\hat{\gamma}_{1}=\left[\delta^{w_{1}}_{g}(x)\delta^{w_{1}}_{g}(x)-\delta^{w_{2}}_{g}(x)\delta^{w_{2}}_{g}(x) \right],\\
&\hat{\gamma}_{2}=[2\delta^{w_{1}}_{g}(x)\delta^{w_{2}}_{g}(x)],\\
&\hat{\gamma}_{x}=[2\delta^{w_{1}}_{g}(x)\delta^{w_{3}}_{g}(x)],\\
&\hat{\gamma}_{y}=[2\delta^{w_{2}}_{g}(x)\delta^{w_{3}}_{g}(x)],\\
&\hat{\gamma}_{z}=[2\delta^{w_{3}}_{g}(x)\delta^{w_{3}}_{g}(x)
-\delta^{w_{1}}_{g}(x)\delta^{w_{1}}_{g}(x)
-\delta^{w_{2}}_{g}(x)\delta^{w_{2}}_{g}(x)]/3,
\end{split}
\end{equation}
where $\delta^{w_{i}}_{g}(x)$ is a filtered density fields,and $i$ indicates $\hat{x},\hat{y},\hat{z}$ directions. In Fourier space $w_{i}$ is given by
\begin{equation}
\delta^{w_{i}}_{g}(\mathbf{k})=\delta_{g}(\mathbf{k})w_{i}(\mathbf{k}).
\end{equation}
Here $\delta_{g}$ is the Gaussianized density field and $w_{i}(\mathbf{k})$ is a optimal filter, defined as 
\begin{equation}
w_{i}(\mathbf{k})=\ii \hat{k}_{i} \left[\frac{P(k)f(k)}{QP^{2}_{tot}(k)}\right]^{1/2},
\label{equ:w}
\end{equation}
with 
\begin{equation}
Q=\int \frac{2k^{2}dk}{15\pi^{2}}\frac{P^{2}(k)}{P^{2}_{tot}(k)}f^{2}(k)
\label{equ:Q}
\end{equation}
and
\begin{equation}
P_{tot}(k)=P(k)+P_{N}(k)
 \end{equation}
 is the observed power spectrum which includes both signal and noise. The filter $f(k)$ is from Eq. \eqref{equ:f}.\\
 
 Then the reconstructed 3D convergence field is
 \begin{equation}
 \label{equ:kappa3D}
 \begin{split}
 \kappa_{3\mathrm{D}}(\mathbf{k})=
 \frac{1}{k^{2}}
 [
&(k_{1}^{2}-k_{2}^{2})\gamma_{1}(\mathbf{k})
 +2k_{1}k_{2}\gamma_{2}(\mathbf{k}) 
 +2k_{1}k_{3}\gamma_{x}(\mathbf{k})\\
&+2k_{2}k_{3}\gamma_{y}(\mathbf{k})
 +(2k_{3}^2-k_1^2-k_2^2)\gamma_{z}(\mathbf{k})
 ].
 \end{split}
 \end{equation}
%========wiener of kappa================================================
To correct the bias and reduce the noise in reconstructed field,
 we write the reconstructed clean field $\hat{\kappa}$ as 
\begin{equation}
\label{eq: kappa}
\hat{\kappa}(k_{\perp},k_{\parallel})=\frac{\kappa_{3\mathrm{D}}(k_{\perp},k_{\parallel})}{b(k_{\perp},k_{\parallel})} W(k_{\perp},k_{\parallel}),
\end{equation}
where bias factor 
\begin{equation}
b(k_{\perp},k_{\parallel})=\frac{P_{\kappa_{3\mathrm{D}}\delta}(k_{\perp},k_{\parallel})}{P_{\delta}(k_{\perp},k_{\parallel})}
\label{equ:bias}
\end{equation}
 and Wiener filter 
\begin{equation}
W(k_{\perp},k_{\parallel})=\frac{P_{\delta}(k_{\perp},k_{\parallel})}{P_{\kappa_{3\mathrm{D}}}(k_{\perp},k_{\parallel})/b^{2}(k_{\perp},k_{\parallel})}.
\end{equation}
Here, the noise power spectrum is 
\begin{equation}
P_{n}(k_{\perp},k_{\parallel})=P_{\kappa 3\mathrm{D}}(k_{\perp},k_{\parallel})-b^2(k_{\perp},k_{\parallel})P_{\delta}(k_{\perp},k_{\parallel}).
\end{equation}


\section{\label{sec:performance}Performance of reconstruction}
%We have discussed the tidal reconstruction with local dark matter 
%density field with 2D and 3D method \citep{2015:zhu}. 
%In practice, however, we can only observe galaxies instead of dark matter
% density fields. And the distribution of halo is a biased tracer of the
%  underlying dark matter density. Therefore, the reconstruction in view of
% halo is even more relevant to the observation.\\

In this section we perform reconstruction in simulated halos fields,
following almost the same process in previous papers. 
To better understand the discrepancy between reconstruction in dark matter and
halos, we discuss the bias and noise of halos, which tell us how to decrease
the halos noise in reconstruction. And then, we also use the 
cross-correlation coefficient to quantify the result of reconstruction 
and discuss the bias and noise of it.\\

%==============================================================
\subsection{Simulation}
{\color{red}(*)}We run N-body simulations using the $\mathrm{CUBEP^3M}$ code with $2048^3$
 dark matter particles in a box of side length $L=1.2\ \mathrm{Gpc}/h$.
  We have
 adopted the following set of cosmological parameter values:
 $\Omega_{b}=0.049$, $\Omega_{c}=0.259$, $h=0.678$, 
 $A_s=2.139\times 10^{-9}$ and $n_s=0.968$.
 Ten runs with independent initial conditions were performed to provide
  better statistics. In the following calculations we use outputs 
  at $z=0$. \\
\subsection{\label{property_h}Properties of halos fields}
In this subsection, we choose three different number-density
($0.0048\ h^{3}\mathrm{Mpc}^{-3}$, 
$0.0024\ h^{3}\mathrm{Mpc}^{-3}$ and $0.0012\ h^{3}\mathrm{Mpc}^{-3}$)
 halos fields to show the properties of halos fields.
{\color{red} here, describe how to get these halos fields with different 
number density and the mass of each halos field. And it may be added in subsection of Simulation} \\

%we show the bias (without shot noise)
%power spectra of simulated
%halos fields $P_{hh}$.
% To understand better, we also show the bias (without shot noise) 
% of halos field 
%$b(k)=P_{\delta h}(k)/P_{\delta\delta}(k)$, and the noise power 
%$P_n(k)/b^{2}(k)=P_{hh}(k)/b^2(k)-P_{\delta \delta}(k)$, 
%defined via the stochasticity between halos and the dark matter.

%============halos property==Figure added here==========================
\begin{figure}[tbp]
\begin{center}
\includegraphics[width=0.48\textwidth]{../../eps/Sim_halo_property_bias.eps}
\end{center}
%\vspace{-0.7cm}
\caption{Halo biases of different number-density halos fields.
The dashed, the dash-dotted and the dotted line shows the bias of halos
 fields with different number density. And linear biases (horizontal solid
  lines) are  obtained by averaging the first 6 k-bins 
  ($k< 0.04\ h/\mathrm{Mpc}$) for each line. }
\label{fig:Sim_bias}
\end{figure}

\begin{figure}[tbp]
\begin{center}
\includegraphics[width=0.48\textwidth]{../../eps/Sim_halo_property_PS.eps}
\end{center}
%\vspace{-0.7cm}
\caption{Top panel: the auto-power spectrum of halo fields. The 
solid lines are the power spectrum of halo fields with number density of
$0.0048\ h^{3}\mathrm{Mpc}^{-3}$, $0.0036\ h^{3}\mathrm{Mpc}^{-3}$, 
$0.0024\ h^{3}\mathrm{Mpc}^{-3}$ and $0.0012\ h^{3}\mathrm{Mpc}^{-3}$,
 respectively.
The horizontal dot dash lines are corresponding shot noise $1/\bar{n}$ 
for every halo fields. Here $\bar{n}$ is the number density.
The black dashed line is the power
spectrum of dark matter field.
Bottom panel: same as the top panel, but the colored solid lines are the 
matter-halo cross power spectrum.}
\label{fig:Sim_PS}
\end{figure}



\begin{figure}[tbp]
\begin{center}
%\includegraphics[width=0.48\textwidth]{../../eps/Sim_halo_property_CC.eps}
\includegraphics[width=0.48\textwidth]{../../eps/Sim_halo_property_Pn.eps}
\end{center}
%\vspace{-0.7cm}
\caption{Top panel: the auto-power spectrum of halo fields. The 
solid lines are the power spectrum of halo fields with number density of
$0.0048\ h^{3}\mathrm{Mpc}^{-3}$, $0.0036\ h^{3}\mathrm{Mpc}^{-3}$, 
$0.0024\ h^{3}\mathrm{Mpc}^{-3}$ and $0.0012\ h^{3}\mathrm{Mpc}^{-3}$,
 respectively.
The horizontal dot dash lines are corresponding shot noise $1/\bar{n}$ 
for every halo fields. Here $\bar{n}$ is the number density.
The black dashed line is the power
spectrum of dark matter field.
Bottom panel: same as the top panel, but the colored solid lines are the 
matter-halo cross power spectrum.}
\label{fig:Sim_CC_Pn}
\end{figure}
%==============================================================
In Fig. \ref{fig:Sim_bias}, we show the bias factor  
 of simulated halos fields in three different number densities. 
 Here we define the halos
 bias factor as $b_{h}(k)=P_{\delta h}(k)/P_{\delta\delta}(k)$,
 which without the shot noise $1/\bar{n}$.
 Here, the $P_{\delta \delta}(k)$ and $P_{\delta h}$ are the auto-correlation 
 power
 spectrum of dark matter and the halo-matter cross-correlation power spectrum. 
 The dashed, the dash-dotted and the dotted lines show the bias of halos
 fields with number density $0.0012\ h^{3}\mathrm{Mpc}^{-3}$,
 $0.0024\ h^{3}\mathrm{Mpc}^{-3}$ and $0.0048\ h^{3}\mathrm{Mpc}^{-3}$,
 respectively. The solid line with corresponding color is the linear bias
 obtained by averaging the first 6 $k$-bins ($k< 0.04\ h/\mathrm{Mpc}$)
 of $b_{h}(k)$.
 On small scales, the bias is no longer a scale-independent factor. 
 In this paper, every point is averaged all ten simulations and the error
 is given by the bootstrap.\\
 
 The shot noise and stochasticity of halos are always an important and
  complicated topic in relevant studies {\color{blue} cite here}.
  A batter understanding about the shot noise and 
 stochasticity is helpful to decrease its influence in our reconstruction
 (in this paper, a wiener filter is applied to suppress the noise in halos
 fields, which we discuss this further below).
 The shot noise, assumed roughly to $1/\bar{n}$ here, can be subtract from
 the halos auto-cross power spectrum.
 {\color{blue} (Although some papers suggest the shot-noise subtraction is not safely.. may cite here?)} 
 And note that the matter power spectrum 
 dose not require shot noise subtraction because of the large number of 
 dark matter particles. In Fig. \ref{fig:Sim_PS}, we plot the power spectra
 of three different number-density halos fields. To compare these power
 spectra with dark matter power spectra, halos 
 shot-noise subtracted auto-correlation
 power spectra is divided by square of the bias factor 
   $P_{hh}(k)/b^2(k)$ (here, the bias we use is the scale-dependent bias and
   see Fig. \ref{fig:Sim_bias}). 
 Here, the black solid line is the matter power spectrum and coloured 
 lines indicate the halos auto-correlation power spectra of 
 different number density
 as in Fig. \ref{fig:Sim_bias}. We find, on large scales, different power
 spectra are almost have a same value. Which suggest 
  even with only shot-noise subtraction, the halos power spectra
 can trace the dark matter power well (less relative error) on large scales
 in choosed number density and halo mass. However, on 
 small scale, the halos power spectra are always above the dark matter power
 even applied the shot-noise subtraction, and higher number density with
 larger difference. So if we want to quantify the noise
 of halos fields, consider the
 shot noise only is not enough, especially on small scale.\\
 
 Understanding everything about the shot noise and the stochasticity is too
 complex and difficult, but it worth the effort. 
 However, we do not need to do it in our reconstruction. 
 Here, we define the noise power of halos field
\begin{equation}
P_{n}(k)=P_{hh}(k)-b^{2}_{h}(k)P_{\delta\delta}(k),
\label{eq:noise_halo}
\end{equation}
in which both the shot noise and the stochasticity have contribution.
By this equation, we can calculate the noise of halos from simulations.
In Fig. \ref{fig:Sim_CC_Pn}, we plot the $P_{n}(k)/b_{h}^{2}(k)$ 
to compare with
the dark matter power spectrum (the black solid line).
Identical to \ref{fig:Sim_PS}, we use the dashed, the dash-dotted 
and the dotted line show the noise power of different number density.
Three thin coloured solid lines indicate the shot noise of corresponding number density, which have been divided by $b^2_{h}(k)$ too. 
In this way, we can compare the shot noise $1/\bar{n}$, the noise of 
halos $P_{n}(k)$ and the power of dark matter $P_{\delta\delta}(k)$ directly.
On the linear scales ($k<0.1\ h/\mathrm{Mpc}$), all the noise lines is over
the corresponding shot noise line, which is because of stochasticity.
And comparing to the shot noise, the noise power $P_{n}$ is almost have 
the same order, which meaning the noise is dominated by stochasticity
on large scale in choosed number density and halo mass.
 In nonlinear region, the nonlinear
evolution involve much complex effects, we do not discuss it here.
The correlation coefficient of halos and dark matter,
which defined as $r_{\mathrm{h}}=P_{\delta h}/\sqrt{P_{\delta\delta}P_{hh}}$, is always
used to quantify the relation between the distribution of halo and matter.
 From Eq. \ref{eq:noise_halo} and the definition of $b_{h}(k)$ we can prove 
$\frac{P_{n}/b_{h}^2}{P_{\delta\delta}}=\frac{1}{r_{\mathrm{h}}^2}-1$,
so we do not show the correlation coefficient here.
\\

By these discussion about the bias, the power and the noise of halos
fields, we hope to find the differences between the 
reconstruction in halos and dark matter, which we discuss in subsection
 \ref{perf_of_recon}.
%==============================================================
%\begin{figure}[tbp]
%\begin{center}
%\includegraphics[width=0.48\textwidth]{../../eps/Sim_bias_errorbar.eps}
%\end{center}
%%\vspace{-0.7cm}
%\caption{Bias of simulated halo fields. Here we define the bias as 
%$b=\frac{P_{\delta h}}{P_{\delta}}$ and obtain a constant bias by
%averaging the first 6 $k$-bins where $k<0.04\ h/\mathrm{Mpc}$. The constant
%bias is $1.1137$, $0.9704$, $0.8986$ and $0.8456$ for the halo fields with
%number density 
%$0.0012\ (h/\mathrm{Mpc})^3$, $0.0024\ (h/\mathrm{Mpc})^3$, 
%$0.0036\ (h/\mathrm{Mpc})^3$ and $0.0048\ (h/\mathrm{Mpc})^3$, respectively.}
%\label{fig:Sim_bias}
%\end{figure}
%==============================================================
%\begin{figure}[tbp]
%\begin{center}
%\includegraphics[width=0.48\textwidth]{../../eps/Sim_Pn_errorbar.eps}
%\includegraphics[width=0.48\textwidth]{../../eps/Sim_noise_errorbar.eps}
%\end{center}
%%\vspace{-0.7cm}
%\caption{Noise of halo fields for $4$ number density. The top panel we show
%$P_{h}-b^2P_{\delta}$, here $b$ is the $k$-dependent bias 
%$P_{\delta h}/P_{h}$. The correlative horizontal dot dash lines are the 
%shot noise $1/\bar{n}$ of different number density field.
%The bottom panel is inverse signal to noise ratio 
%$(P_{h}-b^2P_{\delta})/b^2P_{\delta}$.}
%\label{fig:Sim_noise}
%\end{figure}
%==============================================================
%copyed from zhu,maybe add something by YuYu.
%==============================================================
\subsection{\label{perf_of_recon}Performance of reconstruction }
The performance of reconstruction in halos mainly following the previous works \citep{2012arXiv1202.5804P,2016PhRvD..93j3504Z}. 
However, there are some changes because
of the differences between the dark matter and halo fields. 
Note that because we are using the halos fields to trace dark matter 
distribution, the $\delta(x)$ used here is the halos density field 
({\color{red} explain how to get halos density field in Simulation or it should be another name?})
which have deconvolved bias $b_{h}$.
The algorithm of the reconstruction in halos is as follows.\\
%============================================
(i). Applying a Gaussian smoothing kernel to $\delta(x)$:
\begin{equation}
\bar{\delta}(\mathbf{x})=\int d^{3}x'\mathbf{S}(\mathbf{x}-\mathbf{x}')\delta	(\mathbf{x}'),
\label{equ:smooth kernel}
\end{equation}
where $\mathbf{S}(r)=\mathrm{e}^{-r^{2}/2R^{2}}$ and $R$ is the smoothing
scale. Here, we calculated a lots
of different smoothing scales and find we can obtain the optimal 
reconstruction result with the smoothing scale $R=1.0\ \mathrm{Mpc}/h$. 
So we select $R=1.0\ \mathrm{Mpc}/h$ as the smoothing scale in this paper. 
By this step, we get a smoothed
field $\bar{\delta}_{h}(x)$ which {\color{blue}is considered as} the
cosmic density field and have been suppressed the vary small-scale 
fluctuations.\\
{\color{red} discuss why different number-density halos reconstruction
with the same smoothing scale in Sec. Discussion.}\\
%============================================
(ii). Convolving the smoothed density field 
$\bar{\delta}_{h}(x)$ with a wiener filter 
{\color{red} (about the Wiener and $\delta^{w_{i}}$ should be discussed in 
Sec. Discussion: why we separate the Wiener from filter $w_i$.)}
\begin{equation}
W_{h}(k)=\frac{P_{hh}(k)-P_{n}(k)}{P_{hh}(k)}
=\frac{b_{h}^2(k)P_{\delta\delta}(k)}{P_{hh}(k)},
\label{eq:nbar_filter}
\end{equation}
where $P_{hh}(k)$ is the without shot-noise subtracted halos power spectrum,
$P_{\delta\delta}(k)$ the dark matter power spectrum, $b_{h}(k)$
the bias factor of halos and $P_{n}(k)$ the noise power from 
Eq. \ref{eq:noise_halo}.\\
The Wiener filter is help to suppress the noise of halos fields if we know
the signal and noise accurately. However, in real data, there is no 
approach to distinguish the signal and noise exactly, 
but we can in simulation. 
So the Wiener filter should be calculated in simulations.\\
The halos bias factor $b_h(k)$ used here is a scale-dependent bias.
However, a linear bias factor is usually believed more credible and can
be obtained from observation. In the Sec. \ref{sec:discussion}
{\color{red} here cite the figure}
, we will show that it is no influence on the reconstruction whether 
the scale-dependent (the curved lines in Fig. \ref{fig:Sim_bias}) or 
the scale-independent (the horizontal solid lines in Fig. \ref{fig:Sim_bias}) 
bias is used here.
\\
%=====
 After applying Eq. \eqref{eq:nbar_filter}, we
 suppress the noise in halo field (both the shot noise and stochasticity)
and obtain a smoothed and filtered halo density field $\tilde{\delta}(x)$
 to reconstruction.
\\
%============================================
(iii).{\color{red}*}
 Then, we convolve the smoothed and filtered density field 
$\tilde{\delta}(x)$ with $w_i$ (from Eq. \ref{equ:w})
 and calculate $\gamma_1$, $\gamma_2$, $\gamma_x$,
$\gamma_y$ and $\gamma_z$ by 
Eqs. \eqref{equ:gamma}\eqref{equ:w}\eqref{equ:Q}. And by
Eq. \eqref{equ:kappa3D} we obtain 3D convergence field 
$\kappa(\bm{k})$.\\
%============================================
(iv).{\color{red}*}
 By Eq. (\ref{eq:kappa}), we correct the bias and suppress the noise
of the estimated convergence field $\hat{\kappa}$. Here,
 $b(k_{\perp},k_{\parallel})$ and $W(k_{\perp},k_{\parallel})$ 
 are calculated by averaging in all ten simulations.
 {\color{blue} need discuss how to get reconstructed large-scale $\delta_{L}$
 from $\hat{\kappa}$?}\\
 
 The tidal shear estimator is an optimal estimator  under the Gaussian
  assumption, which have discussed in  Ref. \citep{2016PhRvD..93j3504Z}.
 However, we do not apply a Gaussianization method here. This is because of 
 the following points.
 Firstly, different from the dark matter density field, halos counts have 
 not such high fluctuations on small scale. So it is not necessary to
 Gaussianize to suppress the high weights of the high density regions 
 because of quadratic estimator. Secondly, for halos-counts fields, there are
 many grids without halos, which makes the Gaussianization method 
 lead into some other problems.  
 \\
%===================hereafter: result ====================================

\section{\label{sec:cross_term}The cross term of $\gamma$}
%\section{The cross term of $\gamma$}
We can estimate the $\gamma$ shear from Eq. \eqref{equ:gamma}. If we separate
halo field $\delta$ into two mass bins, such as
\begin{equation}
\delta=\delta_a + \delta_b,
\label{equ:h_d}
\end{equation}
where subscript $a$ and $b$ indicate two mass bins with the same number density.
Obviously, we can obtain $\gamma_{i}^{a}$ and $\gamma_{i}^{b}$ by 
Eq. \eqref{equ:gamma}. Here index $i$ indicates 
'$1$', '$2$', '$x$', '$y$', '$z$'.  In addition, 
we can also estimate the $\gamma$ field cross $\delta_{a}$ and $\delta_{b}$.\\
 The cross term of $\gamma$ can is
\begin{equation}
\label{equ:gamma_high}
\begin{split}
&\hat{\gamma}_{1}=\left[\delta^{w_{1}}_{a}(x)\delta^{w_{1}}_{b}(x)-\delta^{w_{2}}_{a}(x)\delta^{w_{2}}_{b}(x) \right],\\
&\hat{\gamma}_{2}=[\delta^{w_{1}}_{a}(x)\delta^{w_{2}}_{b}(x)
+\delta^{w_{1}}_{b}(x)\delta^{w_{2}}_{a}(x)],\\
&\hat{\gamma}_{x}=[\delta^{w_{1}}_{a}(x)\delta^{w_{3}}_{b}(x)
+\delta^{w_{1}}_{b}(x)\delta^{w_{3}}_{a}(x)],\\
&\hat{\gamma}_{y}=[\delta^{w_{2}}_{a}(x)\delta^{w_{3}}_{b}(x)
+\delta^{w_{2}}_{b}(x)\delta^{w_{3}}_{a}(x)],\\
&\hat{\gamma}_{z}=[(2\delta^{w_{3}}_{a}(x)\delta^{w_{3}}_{b}(x)
-\delta^{w_{1}}_{a}(x)\delta^{w_{1}}_{b}(x)
-\delta^{w_{2}}_{a}(x)\delta^{w_{2}}_{b}(x)]/3,
\end{split}
\end{equation}
and the final reconstructed $\kappa$ field is 
\begin{equation}
\kappa=\kappa_{a}+\kappa_{b}+2\kappa_{ab}
\label{equ:kappa_d}
\end{equation}
%=============================================================================
\begin{figure}[tbp]
\begin{center}
\includegraphics[width=0.48\textwidth]{../../eps/high_order.eps}
\end{center}
\vspace{-0.7cm}
\caption{Reconstruction by Eq. \eqref{equ:kappa_d}. We separate 
the $\bar{n}=0.0003\ h^{3}\mathrm{Mpc}^{-3}$ halos field into two mass bins $\delta_a$
and $\delta_{b}$ and reconstruct the $\kappa$ field.}
\label{fig:H_CC}
%=============================================================================
\end{figure}
\begin{figure}[tbp]
\begin{center}
\includegraphics[width=0.48\textwidth]{../../eps/high_order_24.eps}
\end{center}
\vspace{-0.7cm}
\caption{Identical to Fig. \ref{fig:H_CC}, but the number density of separated
halo field is $0.0024\ h^{3}\mathrm{Mpc}^{-3}$.}
\label{fig:H_CC_24}
\end{figure}
%=============================================================================
% If you have acknowledgments, this puts in the proper section head.
%\begin{acknowledgments}
% put your acknowledgments here.
%\end{acknowledgments}

% Create the reference section using BibTeX:
\bibliography{./tides.bib}
\bibliographystyle{apsrev}
\end{document}
%
% ****** End of file apstemplate.tex ******

