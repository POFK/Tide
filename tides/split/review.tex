%\begin{equation}
%a+n S=4
%\label{equ:eq.2}
%\end{equation}
%cite as Eq.(\ref{equ:eq.2})
The basic idea of cosmic tidal reconstruction is presented originally in 
Ref. \citep{2012:pen} and detailed description can be found in 
Ref. \citep{2015:zhu}.
In this section we will briefly review it.\\

%=============================================================================
Not only the small-scale density fluctuation but also the long-wavelength
 tidal field influence the evolution of local small-scale density field. So
 we can use small scale observations to solve the large scale tidal shear 
 and gravitational potential. If we just assume the leading-order coupling
 between the long-wavelength tidal field and small-scale density field, 
  the tidal distortion of the local small-scale power spectra can be 
  written as
\begin{equation}
P(\mathbf{k},\tau)|_{t_{ij}}=P_{1s}(k,\tau)+\hat{k}^{i}\hat{k}^{j}t_{ij}^{(0)}P_{1s}(k,\tau)f(k,\tau),
\label{equ:Pk_s}
\end{equation}
where $\hat{k}$ is the unit vector in the direction of $\mathbf{k}$, 
$P_{1s}$ is the small-scale power spectrum, the long-wavelength 
tidal field is $t_{ij}=\Phi_{L,ij}-\delta_{ij}\nabla^{2}\Phi_{L}/3$ and 
$f(k,\tau)$ are used to describe the coupling of the 
long-wavelength
tidal field and small-scale density field. In this paper, we use superscript
$(0)$ and $\tau$ to denotes the "initial" time and conformal time,
respectively. \\

The filter $f(\mathbf{k},\tau)$ is defined as
\begin{equation}
f(\mathbf{k},\tau)=2\alpha -\beta \frac{d\ln P_{1s}(k,\tau) }{d\ln k},
\label{equ:f}
\end{equation}
where 
\begin{equation}
\begin{split}
&\alpha(\tau)=-D_{\sigma 1}(\tau)+F(\tau),\\
&\beta(\tau)=F(\tau),
\end{split}
\end{equation}
with 
\begin{equation}
D_{\sigma 1}(\tau)=\int^{\tau}_{0}d\tau'\frac{H(\tau)D(\tau')-H{\tau'}D{\tau}}{\dot{H}(\tau')D(\tau')-H(\tau')\dot{D}(\tau')}\frac{T(\tau')D(\tau')}{D(\tau)},
\end{equation}
\begin{equation}
F(\tau)=\int^{\tau}_{0}d\tau''a(\tau'')T(\tau'')\int^{\tau}_{\tau''}d\tau'/a(\tau').
\end{equation}
Here, $D(\tau)$ is the linear growth function, $T(\tau)=D(\tau)/a(\tau)$ is the linear transfer function and $H(\tau)=d\ln a/d\tau$ is the comoving Hubble parameter.\\

From Eq. \eqref{equ:Pk_s}, we express small-scale density fluctuations
with the long-wavelength tidal field $t_{ij}$ and local small-scale
density field $\delta_{1s}$. Then a tidal shear estimator can be constructed
as following.\\

The tidal tensor $t_{ij}$ can be decomposed as
\begin{equation}
t_{ij}=
\begin{pmatrix}
\gamma_{1}-\gamma_{z} & \gamma_{2} & \gamma_{x} \\
\gamma_{2} & -\gamma_{1}-\gamma_{z} & \gamma_{y} \\
\gamma_{x} & \gamma_{y} & 2\gamma_{z}
\end{pmatrix},
\label{tij_c5}
\end{equation}
where
$\gamma_{1}=(\Phi_{L,11}-\Phi_{L,22})/2$,
$\gamma_{2}=\Phi_{L,12}$,
$\gamma_{x}=\Phi_{L,13}$,
$\gamma_{y}=\Phi_{L,23}$,
$\gamma_{z}=(2\Phi_{L,23}-\Phi_{L,11}-\Phi_{L,22})/6$.\\

In 3D reconstruction,
%==================================================================
%\subsection{Tidal Shear Estimators And Density Reconstruction Algorithm}
%The Jacobian matrix can be described the density contrast (Eq. (\ref{equ:Jacobian})) as
% well as the mapping between the source and image planes in gravitational lensing.  %\citep{2008lu}. 
%So we apply CMB lensing techniques to cosmic tidal reconstruction. 
%
%In the long wavelength limit and under the Gaussian assumption, quadratic estimators can be constructed either by using the maximum likelihood method %\citep{2008lu}
%or the inverse variance weighting \citep{2010lu,2012bucher}:
%\begin{equation}
%\hat{\gamma}_1=\frac{1}{Q}\int \frac{d^3k}{(2\pi)^3} \frac{|\delta_g(\mathbf{k})|^2}{L^3}\frac{P(k)}{P^2_{tot}(k)}f(k)(\hat{k}_{1}^{2}-\hat{k}^{2}_{2}),
%\end{equation}
%\begin{equation}
%\hat{\gamma}_2=\frac{1}{Q}\int \frac{d^3k}{(2\pi)^3} \frac{|\delta_g(\mathbf{k})|^2}{L^3}\frac{P(k)}{P^2_{tot}(k)}f(k)(2\hat{k}_{1}\hat{k}_{2}).
%\end{equation}

%If we assume that the long wavelength tidal shear fields vary more slowly than the small scale density field and the tidal shear field is constant in space, the unbiased minimum variance estimates of the spatial varying tidal field in the long wavelength limit is given by
%==================================================================
\begin{equation}
\label{equ:gamma}
\begin{split}
&\hat{\gamma}_{1}=\left[\delta^{w_{1}}_{g}(x)\delta^{w_{1}}_{g}(x)-\delta^{w_{2}}_{g}(x)\delta^{w_{2}}_{g}(x) \right],\\
&\hat{\gamma}_{2}=[2\delta^{w_{1}}_{g}(x)\delta^{w_{2}}_{g}(x)],\\
&\hat{\gamma}_{x}=[2\delta^{w_{1}}_{g}(x)\delta^{w_{3}}_{g}(x)],\\
&\hat{\gamma}_{y}=[2\delta^{w_{2}}_{g}(x)\delta^{w_{3}}_{g}(x)],\\
&\hat{\gamma}_{z}=[2\delta^{w_{3}}_{g}(x)\delta^{w_{3}}_{g}(x)
-\delta^{w_{1}}_{g}(x)\delta^{w_{1}}_{g}(x)
-\delta^{w_{2}}_{g}(x)\delta^{w_{2}}_{g}(x)]/3,
\end{split}
\end{equation}
where $\delta^{w_{i}}_{g}(x)$ is a filtered density fields,and $i$ indicates $\hat{x},\hat{y},\hat{z}$ directions. In Fourier space $w_{i}$ is given by
\begin{equation}
\delta^{w_{i}}_{g}(\mathbf{k})=\delta_{g}(\mathbf{k})w_{i}(\mathbf{k}).
\end{equation}
Here $\delta_{g}$ is the Gaussianized density field and $w_{i}(\mathbf{k})$ is a optimal filter, defined as 
\begin{equation}
w_{i}(\mathbf{k})=\ii \hat{k}_{i} \left[\frac{P(k)f(k)}{QP^{2}_{tot}(k)}\right]^{1/2},
\label{equ:w}
\end{equation}
with 
\begin{equation}
Q=\int \frac{2k^{2}dk}{15\pi^{2}}\frac{P^{2}(k)}{P^{2}_{tot}(k)}f^{2}(k)
\label{equ:Q}
\end{equation}
and
\begin{equation}
P_{tot}(k)=P(k)+P_{N}(k)
 \end{equation}
 is the observed power spectrum which includes both signal and noise. The filter $f(k)$ is from Eq. \eqref{equ:f}.\\
 
 Then the reconstructed 3D convergence field is
 \begin{equation}
 \label{equ:kappa3D}
 \begin{split}
 \kappa_{3\mathrm{D}}(\mathbf{k})=
 \frac{1}{k^{2}}
 [
&(k_{1}^{2}-k_{2}^{2})\gamma_{1}(\mathbf{k})
 +2k_{1}k_{2}\gamma_{2}(\mathbf{k}) 
 +2k_{1}k_{3}\gamma_{x}(\mathbf{k})\\
&+2k_{2}k_{3}\gamma_{y}(\mathbf{k})
 +(2k_{3}^2-k_1^2-k_2^2)\gamma_{z}(\mathbf{k})
 ].
 \end{split}
 \end{equation}
%========wiener of kappa================================================
To correct the bias and reduce the noise in reconstructed field,
 we write the reconstructed clean field $\hat{\kappa}$ as 
\begin{equation}
\label{eq: kappa}
\hat{\kappa}(k_{\perp},k_{\parallel})=\frac{\kappa_{3\mathrm{D}}(k_{\perp},k_{\parallel})}{b(k_{\perp},k_{\parallel})} W(k_{\perp},k_{\parallel}),
\end{equation}
where bias factor 
\begin{equation}
b(k_{\perp},k_{\parallel})=\frac{P_{\kappa_{3\mathrm{D}}\delta}(k_{\perp},k_{\parallel})}{P_{\delta}(k_{\perp},k_{\parallel})}
\label{equ:bias}
\end{equation}
 and Wiener filter 
\begin{equation}
W(k_{\perp},k_{\parallel})=\frac{P_{\delta}(k_{\perp},k_{\parallel})}{P_{\kappa_{3\mathrm{D}}}(k_{\perp},k_{\parallel})/b^{2}(k_{\perp},k_{\parallel})}.
\end{equation}
Here, the noise power spectrum is 
\begin{equation}
P_{n}(k_{\perp},k_{\parallel})=P_{\kappa 3\mathrm{D}}(k_{\perp},k_{\parallel})-b^2(k_{\perp},k_{\parallel})P_{\delta}(k_{\perp},k_{\parallel}).
\end{equation}
