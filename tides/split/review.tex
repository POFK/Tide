%\begin{equation}
%a+n S=4
%\label{equ:eq.2}
%\end{equation}
%cite as Eq.(\ref{equ:eq.2})
The basic idea of cosmic tidal reconstruction is introduced originally in 
Ref. \citep{2012arXiv1202.5804P} and detailed description can be found in 
Ref. \citep{2016PhRvD..93j3504Z}.
In this section we will briefly review it.\\

%=============================================================================
Not only the small-scale density fluctuation but also the long-wavelength
 tidal field influence the evolution of local small-scale density field.
The local anisotropic distortion, which derived from the long-wavelength 
traceless tidal field $t_{ij}=\Phi_{L,ij}-\delta_{ij}\nabla^{2}\Phi_{L}/3$,
where $\Phi_{L}$ is the long-wavelength gravitational potential,
 can be used
to reconstruct the large-scale density field.
 If we just assume the leading-order coupling
 between the long-wavelength tidal field and small-scale density field, 
  the tidal distortion of the local small-scale power spectra can be 
  written as
\begin{equation}
P(\bm{k},\tau)|_{t_{ij}}=P_{1s}(k,\tau)+\hat{k}^{i}\hat{k}^{j}t_{ij}^{(0)}P_{1s}(k,\tau)f(k,\tau),
\label{equ:Pk_s}
\end{equation}
where $\hat{k}$ is the unit vector in the direction of $\bm{k}$, 
$P_{1s}$ is the isotropic linear power spectrum
%the long-wavelength 
%tidal field is $t_{ij}=\Phi_{L,ij}-\delta_{ij}\nabla^{2}\Phi_{L}/3$ 
and $f(k,\tau)$ are used to describe the coupling of the 
long-wavelength
tidal field and small-scale density field. Here, the superscript
$(0)$ and $\tau$ are denote the "initial" time and conformal time,
respectively. \\
Here, the filter $f(\bm{k},\tau)$ is defined as
\begin{equation}
f(k,\tau)=2\alpha -\beta \frac{d\ln P_{1s}(k,\tau) }{d\ln k},
\label{equ:f}
\end{equation}
where 
\begin{equation}
\begin{split}
&\alpha(\tau)=-D_{\sigma 1}(\tau)+F(\tau),\\
&\beta(\tau)=F(\tau),
\end{split}
\end{equation}
with 
\begin{equation}
D_{\sigma 1}(\tau)=\int^{\tau}_{0}d\tau'\frac{H(\tau)D(\tau')-H(\tau')D(\tau)}{\dot{H}(\tau')D(\tau')-H(\tau')\dot{D}(\tau')}\frac{T(\tau')D(\tau')}{D(\tau)},
\end{equation}
\begin{equation}
F(\tau)=\int^{\tau}_{0}d\tau''a(\tau'')T(\tau'')\int^{\tau}_{\tau''}d\tau'/a(\tau').
\end{equation}
Here, $D(\tau)$ is the linear growth factor, $T(\tau)=D(\tau)/a(\tau)$ is the linear transfer function and $H(\tau)=d\ln a/d\tau$ is the comoving Hubble parameter.\\

%From Eq. \eqref{equ:Pk_s}, we express small-scale density fluctuations
%with the long-wavelength tidal field $t_{ij}$ and local small-scale
%density field $\delta_{1s}$. Then a tidal shear estimator can be constructed
%as following.\\
The tensor $\Phi_{L,ij}$ can be written as 
\begin{equation}
\Phi_{L,ij}= \delta_{ij}\kappa +t_{ij},
\end{equation}
where the convergence field $\kappa = \nabla^{2}\Phi_{L}/3$.\\
The traceless tidal tensor $t_{ij}$ can be decomposed as in weak lensing:
\begin{equation}
t_{ij}=
\begin{pmatrix}
\gamma_{1}-\gamma_{z} & \gamma_{2} & \gamma_{x} \\
\gamma_{2} & -\gamma_{1}-\gamma_{z} & \gamma_{y} \\
\gamma_{x} & \gamma_{y} & 2\gamma_{z}
\end{pmatrix},
\label{tij_c5}
\end{equation}
where
$\gamma_{1}=(\Phi_{L,11}-\Phi_{L,22})/2$,
$\gamma_{2}=\Phi_{L,12}$,
$\gamma_{x}=\Phi_{L,13}$,
$\gamma_{y}=\Phi_{L,23}$,
$\gamma_{z}=(2\Phi_{L,23}-\Phi_{L,11}-\Phi_{L,22})/6$.\\

And under the Gaussian assumption and in the long-wavelength limit,
the optimal tidal shear estimators can be given as
(more details can found in Ref. \citep{2016PhRvD..93j3504Z})
%==================================================================
%\subsection{Tidal Shear Estimators And Density Reconstruction Algorithm}
%The Jacobian matrix can be described the density contrast (Eq. (\ref{equ:Jacobian})) as
% well as the mapping between the source and image planes in gravitational lensing.  %\citep{2008lu}. 
%So we apply CMB lensing techniques to cosmic tidal reconstruction. 
%
%In the long wavelength limit and under the Gaussian assumption, quadratic estimators can be constructed either by using the maximum likelihood method %\citep{2008lu}
%or the inverse variance weighting \citep{2010lu,2012bucher}:
%\begin{equation}
%\hat{\gamma}_1=\frac{1}{Q}\int \frac{d^3k}{(2\pi)^3} \frac{|\delta_g(\bm{k})|^2}{L^3}\frac{P(k)}{P^2_{tot}(k)}f(k)(\hat{k}_{1}^{2}-\hat{k}^{2}_{2}),
%\end{equation}
%\begin{equation}
%\hat{\gamma}_2=\frac{1}{Q}\int \frac{d^3k}{(2\pi)^3} \frac{|\delta_g(\bm{k})|^2}{L^3}\frac{P(k)}{P^2_{tot}(k)}f(k)(2\hat{k}_{1}\hat{k}_{2}).
%\end{equation}

%If we assume that the long wavelength tidal shear fields vary more slowly than the small scale density field and the tidal shear field is constant in space, the unbiased minimum variance estimates of the spatial varying tidal field in the long wavelength limit is given by
%==================================================================
\begin{equation}
\label{equ:gamma}
\begin{split}
&\hat{\gamma}_{1}=\left[\delta^{w_{1}}(x)\delta^{w_{1}}(x)-\delta^{w_{2}}(x)\delta^{w_{2}}(x) \right],\\
&\hat{\gamma}_{2}=[2\delta^{w_{1}}(x)\delta^{w_{2}}(x)],\\
&\hat{\gamma}_{x}=[2\delta^{w_{1}}(x)\delta^{w_{3}}(x)],\\
&\hat{\gamma}_{y}=[2\delta^{w_{2}}(x)\delta^{w_{3}}(x)],\\
&\hat{\gamma}_{z}=[2\delta^{w_{3}}(x)\delta^{w_{3}}(x)
-\delta^{w_{1}}(x)\delta^{w_{1}}(x)
-\delta^{w_{2}}(x)\delta^{w_{2}}(x)]/3,
\end{split}
\end{equation}
where $\delta^{w_{i}}(x)$ are filtered density fields
\begin{equation}
\delta^{w_{i}}(\bm{k})=\delta(\bm{k})w_{i}(\bm{k}).
\end{equation}
 and $i=1,2,3$ indicates $\hat{x}$, $\hat{y}$ and $\hat{z}$ directions.
In the dark matter case, a Gaussianization method is needed because the 
Gaussian assumption. However, for halo reconstruction, the Gaussianization
is not performed for some reasons, which we will discuss in Sec. \ref{sec:performance}.
 In Fourier space, $w_{i}(\bm{k})$ is given by
\begin{equation}
w_{i}(\bm{k})=\ii \hat{k}_{i} \left[\frac{P(k)f(k)}{QP^{2}_{tot}(k)}\right]^{1/2},
\label{equ:w}
\end{equation}
with 
\begin{equation}
Q=\int \frac{2k^{2}dk}{15\pi^{2}}\frac{P^{2}(k)}{P^{2}_{tot}(k)}f^{2}(k).
\label{equ:Q}
\end{equation}
Here, $f(k)$ is from Eq. \eqref{equ:f} and more details will also discussed in
Sec. \ref{sec:performance}.

%\begin{equation}
%P_{tot}(k)=P(k)+P_{N}(k)
% \end{equation}
% is the observed power spectrum which includes both signal and noise. The filter $f(k)$ is from Eq. \eqref{equ:f}.\\
Then, the convergence term $\kappa$ can be estimated from shear terms 
$\gamma_{i}$ 
 \begin{equation}
 \label{equ:kappa3D}
 \begin{split}
 \kappa(\bm{k})=
 \frac{1}{k^{2}}
 [
&(k_{1}^{2}-k_{2}^{2})\gamma_{1}(\bm{k})
 +2k_{1}k_{2}\gamma_{2}(\bm{k}) 
 +2k_{1}k_{3}\gamma_{x}(\bm{k})\\
&+2k_{2}k_{3}\gamma_{y}(\bm{k})
 +(2k_{3}^2-k_1^2-k_2^2)\gamma_{z}(\bm{k})
 ],
 \end{split}
 \end{equation}
and the large-scale density field can be given by the convergence field
because of the Poisson equation.\\
%========wiener of kappa================================================
To correct the bias and reduce the noise in reconstructed field,
 we write the reconstructed clean field $\hat{\kappa}$ as 
\begin{equation}
\label{eq:kappa}
\hat{\kappa}(k_{\perp},k_{\parallel})=\frac{\kappa(k_{\perp},k_{\parallel})}{b(k_{\perp},k_{\parallel})} W(k_{\perp},k_{\parallel}),
\end{equation}
where bias factor 
\begin{equation}
b(k_{\perp},k_{\parallel})=\frac{P_{\kappa\delta}(k_{\perp},k_{\parallel})}{P_{\delta\delta}(k_{\perp},k_{\parallel})}
\label{equ:bias}
\end{equation}
 and Wiener filter 
\begin{equation}
W(k_{\perp},k_{\parallel})=\frac{P_{\delta\delta}(k_{\perp},k_{\parallel})}{P_{\kappa\kappa}(k_{\perp},k_{\parallel})/b^{2}(k_{\perp},k_{\parallel})}.
\end{equation}
Here, the noise power spectrum is 
\begin{equation}
P_{n}(k_{\perp},k_{\parallel})=P_{\kappa 3\mathrm{D}}(k_{\perp},k_{\parallel})-b^2(k_{\perp},k_{\parallel})P_{\delta\delta}(k_{\perp},k_{\parallel}).
\end{equation}
