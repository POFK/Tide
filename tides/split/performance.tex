We have discussed the tidal reconstruction with local dark matter 
density field with 2D and 3D method \citep{2015:zhu}. 
In practice, however, we can only observe galaxies instead of dark matter
 density fields. And the distribution of halo is a biased tracer of the
  underlying dark matter density. Therefore, the reconstruction in view of
 halo is even more relevant to the observation.\\

In this section we perform reconstruction in simulated halos fields,
following the same process in the previous paper. And we also 
use the cross correlation coefficient to quantify the result
of reconstruction .\\

%==============================================================
\subsection{Simulation}
We run N-body simulations using the $\mathrm{CUBEP^3M}$ code with $2048^3$
 dark matter particles in a box of side length $L=1.2\ \mathrm{Gpc}/h$.
  We have
 adopted the following set of cosmological parameter values:
 $\Omega_{b}=0.049$, $\Omega_{c}=0.259$, $h=0.678$, 
 $A_s=2.139\times 10^{-9}$ and $n_s=0.968$.
 Ten runs with independent initial conditions were performed to provide
  better statistics. In the following calculations we use outputs 
  at $z=0$. \\
  
As we have shown before, the purpose of tidal reconstruction is trying to
reconstruct the large-scale tidal field from local small-scale density
field. In the following of this subsection, we show the shot noise and
the inverse signal to noise ratio (N/S) to help us understand that
how much information we can obtain from different halo fields.\\
%============simulation property==================================
\begin{figure}[tbp]
\begin{center}
\includegraphics[width=0.48\textwidth]{../../eps/Sim_PS_HH_errorbar.eps}
\includegraphics[width=0.48\textwidth]{../../eps/Sim_PS_DH_errorbar.eps}
\end{center}
%\vspace{-0.7cm}
\caption{Top panel: the auto-power spectrum of halo fields. The 
solid lines are the power spectrum of halo fields with number density of
$0.0048\ h^{3}\mathrm{Mpc}^{-3}$, $0.0036\ h^{3}\mathrm{Mpc}^{-3}$, 
$0.0024\ h^{3}\mathrm{Mpc}^{-3}$ and $0.0012\ h^{3}\mathrm{Mpc}^{-3}$,
 respectively.
The horizontal dot dash lines are corresponding shot noise $1/\bar{n}$ 
for every halo fields. Here $\bar{n}$ is the number density.
The black dashed line is the power
spectrum of dark matter field.
Bottom panel: same as the top panel, but the colored solid lines are the 
matter-halo cross power spectrum.}
\label{fig:Sim_PS}
\end{figure}
%==============================================================
%\begin{figure}[tbp]
%\begin{center}
%\includegraphics[width=0.48\textwidth]{../../eps/Sim_bias_errorbar.eps}
%\end{center}
%%\vspace{-0.7cm}
%\caption{Bias of simulated halo fields. Here we define the bias as 
%$b=\frac{P_{\delta h}}{P_{\delta}}$ and obtain a constant bias by
%averaging the first 6 $k$-bins where $k<0.04\ h/\mathrm{Mpc}$. The constant
%bias is $1.1137$, $0.9704$, $0.8986$ and $0.8456$ for the halo fields with
%number density 
%$0.0012\ (h/\mathrm{Mpc})^3$, $0.0024\ (h/\mathrm{Mpc})^3$, 
%$0.0036\ (h/\mathrm{Mpc})^3$ and $0.0048\ (h/\mathrm{Mpc})^3$, respectively.}
%\label{fig:Sim_bias}
%\end{figure}
%==============================================================
%\begin{figure}[tbp]
%\begin{center}
%\includegraphics[width=0.48\textwidth]{../../eps/Sim_Pn_errorbar.eps}
%\includegraphics[width=0.48\textwidth]{../../eps/Sim_noise_errorbar.eps}
%\end{center}
%%\vspace{-0.7cm}
%\caption{Noise of halo fields for $4$ number density. The top panel we show
%$P_{h}-b^2P_{\delta}$, here $b$ is the $k$-dependent bias 
%$P_{\delta h}/P_{h}$. The correlative horizontal dot dash lines are the 
%shot noise $1/\bar{n}$ of different number density field.
%The bottom panel is inverse signal to noise ratio 
%$(P_{h}-b^2P_{\delta})/b^2P_{\delta}$.}
%\label{fig:Sim_noise}
%\end{figure}
%==============================================================
%copyed from zhu,maybe add something by YuYu.
%==============================================================
\subsection{Reconstruction}
The performance of reconstruction we mainly follow our previous works \citep{2012:pen,2015:zhu}. However, there are some changes because
of the difference between the dark matter and halo fields. The algorithm
of the reconstruction in view of halo is as follows.\\
%============================================
(i). Convolving the halo density field with a wiener filter $W$ defined as 
\begin{equation}
%W(\mathbf{k})=\frac{b^2P_{\delta}(\mathbf{k})}{P_{h}(\mathbf{k})+1/\bar{n}},
W(k)=\frac{P_{\delta}(k)}{(P_{h}(k)+1/\bar{n})/b^2},
\label{eq:nbar_filter}
\end{equation}
where $P_{h}$, $P_{\delta}$ and $\bar{n}$ are the auto-power spectrum of 
halo and 
dark matter fields and the number density of the halo filed. The bias
factor $b$ is a $k$-independent bias and here we calculate it by
averaging the bias in the first six $k$-bins where $k<0.04\ h/\mathrm{Mpc}$.\\
 Here, the top
we use $b^2P_{\delta}$ instead of $P_{h}$ because we consider the signal
is the dark matter density field not the halo field, although we use the halo
 field trace the dark matter density field. After applied 
 Eq. \eqref{eq:nbar_filter}, we
 suppress the noise in halo field (both the shot noise {\color{black} and} ).  
Then we obtain the filtered halo density field $\tilde{\delta}_{h}$.\\
%============================================
(ii). Applying a Gaussian smoothing kernel to smooth filtered halo 
density field:
\begin{equation}
\bar{\delta}(\mathbf{x})=\int d^{3}x'\mathbf{S}(\mathbf{x}-\mathbf{x}')\delta	(\mathbf{x}'),
\label{equ:smooth kernel}
\end{equation}
where $\mathbf{S}(r)=\mathrm{e}^{-r^{2}/2R^{2}}$. Here, we calculated a lots
of different smoothing scale and find we can obtain the optimal 
reconstruction result with the smoothing scale of $R=1.0\mathrm{Mpc}/h$. 
So we select $R=1.0\ \mathrm{Mpc}/h$
as the smoothing scale in this paper. By this step, we get a smoothed
field $\tilde{\delta}_{h}$ as matter density field which will be used in 
later reconstruction and which is suppressed the vary small-scale 
fluctuations.\\
%============================================
(iii). Then, we convolve the filtered and smoothed density field 
$\tilde{\delta}_{h}$ and calculate $\gamma_1$, $\gamma_2$, $\gamma_x$,
$\gamma_y$, $\gamma_z$, by 
Eq. \eqref{equ:gamma}\eqref{equ:w}\eqref{equ:Q}. And from
Eq. \eqref{equ:kappa3D} we obtain 3D convergence field 
$\kappa_{3D}(\mathbf{k})$.\\
%============================================
(iv). To calculate $b(k_{\perp},k_{\parallel})$ and
 $W(k_{\perp},k_{\parallel})$ to correct the estimated $\kappa_{3D}$, we 
 estimate the power spectrum $P_{\kappa_{3D}}$ and $P_{\kappa_{3D}\delta}$ use 
 these ten simulations. Then we obtain the reconstructed 
 $\hat{\kappa(\mathbf{k})}$ from Eq. (\ref{eq: kappa}).\\
 
 In this paper, we remove the step of Gaussianization in our reconstruction.
 This is because there are not such huge fluctuations in halos fields 
 compare to dark matter density fields. We applied Gaussianization smoothing
 kernel in reconstruction with dark matter density field to avoid the effect
 because of the huge small-scale fluctuations. However, in halo fields 
 reconstruction, we obtain a batter result without Gaussianization.\\
 %==============================================================
\subsection{Result}
In this subsection, we show the correlation coefficient, bias and noise of 
reconstruction.\\

The correlation coefficient we defined as 
\begin{equation}
r(k)=\frac{P_{\delta\kappa}(k)}{\sqrt{P_{\delta \delta}(k)P_{\kappa \kappa}(k)}},
\label{equ:r}
\end{equation}
and the bias factor is 
\begin{equation}
b(k)=\frac{P_{\delta \kappa}(k)}{P_{\delta \delta}(k)}
\label{equ:bias}
\end{equation}

%==============================================================
\begin{figure}[tbp]
\begin{center}
\includegraphics[width=0.48\textwidth]{../../eps/recon_PS_errorbar.eps}
\end{center}
%\vspace{-0.7cm}
\caption{Auto-power spectrum of dark matter, halo and reconstructed $\kappa$ 
fields and cross-power spectrum of matter-halo and matter-kappa, 
respectively. Here the  number density of halo fields is 
$0.0024\ h^{3}\mathrm{Mpc}^{-3}$ and the smoothing scale is
 $R=1.0\ \mathrm{Mpc}/h$.}
\label{fig:recon_PS}
\end{figure}
%==============================================================
\begin{figure}[tbp]
\begin{center}
\includegraphics[width=0.48\textwidth]{../../eps/CC_ND_S1.0.eps}
\end{center}
%\vspace{-0.7cm}
\caption{Correlation coefficient of reconstructed tidal field $\kappa$
and dark matter density field $\delta$ (Eq. \eqref{equ:r}). 
Here we choose the number density 
of halos fields as
 $0.0012\ h^{3}\mathrm{Mpc}^{-3}$, $0.0024\ h^{3}\mathrm{Mpc}^{-3}$, 
$0.0036\ h^{3}\mathrm{Mpc}^{-3}$ and $0.0048\ h^{3}\mathrm{Mpc}^{-3}$ 
and apply
a smoothing kernel with $R=1.0\ \mathrm{Mpc}/h$ 
(Eq. \eqref{equ:smooth kernel}).
We also plot the correlation coefficient of the reconstructed field
from halo fields with number density of $0.0003\ h^{3}\mathrm{Mpc}^{-3}$,
 which is 
the average number density of BOSS.}
\label{fig:recon_CC}
\end{figure}
%=============bias correcting ==============================
In Fig. \ref{fig:recon_bias}, we show the bias of reconstruction.
To quantify the bias of our reconstruction by Eq. \eqref{equ:bias}, we need
to calculate the $Q$ factor in Eq. \eqref{equ:Q} and correct the bias of 
halos and matter fields $b_h=P_{\delta h}/P_{\delta \delta}$.\\%(see \eqref{fig:Sim_bias}).\\

Because of the smoothing kernel we used in Eq. \eqref{equ:smooth kernel}, 
the $Q$ factor becomes:
\begin{equation}
Q=\int \frac{2k^{2}dk}{15\pi^{2}}W^{2}(k)S^2(k)f^{2}(k),
\label{equ:Q2}
\end{equation}
where $S(k)$ is the smoothing kernel in $k$ space and $W(k)$ is the 
wiener filter in Eq. \eqref{eq:nbar_filter}. Here, we solve the $Q$ factor
by a interval from the smallest $k$ to the Nyquist frequency  
in our cube of simulation.\\
Then we can obtain the bias of reconstruction
\begin{equation}
b_{\mathrm{rec}}=\frac{P_{\kappa_{\mathrm{3D}}\delta}}{P_{\delta \delta}} 
\frac{1}{b_{0}^{2}Q},
\label{equ:b_recon}
\end{equation}
here, $b_{0}$ is the $k$-independent bias of simulated halo fields. \\

%==============================================================
\begin{figure}[tbp]
\begin{center}
\includegraphics[width=0.48\textwidth]{../../eps/recon_bias_errorbar.eps}
\end{center}
\vspace{-0.7cm}
\caption{Bias factor $b(k)$ of reconstructed $\kappa$ field from different
 number density halo fields (Eq. \eqref{equ:bias}). 
 Here we correct the bias by 
 Eq. \eqref{equ:b_recon} and get the constant bias of reconstruction as 
 $0.795$, $0.725$, $0.640$ and $0.497$ for reconstructed $\kappa$ fields
 from $0.0048\ h^{3}\mathrm{Mpc}^{-3}$, $0.0036\ h^{3}\mathrm{Mpc}^{-3}$, 
$0.0024\ h^{3}\mathrm{Mpc}^{-3}$ and $0.0012\ h^{3}\mathrm{Mpc}^{-3}$,
 respectively.
Here, the constant bias is obtained by averaging 
the bias of the first 6 $k$-bins.}
\label{fig:recon_bias}
\end{figure}
%==============================================================
\begin{figure}[tbp]
\begin{center}
\includegraphics[width=0.48\textwidth]{../../eps/recon_Pn_b2-1D.eps}
\end{center}
\vspace{-0.7cm}
\caption{The noise spectrum of reconstruction. Here we define the noise 
spectrum as $P_{n}(k)=P_{\kappa}(k)/b^2(k)-P_{\delta}(k)$ and the bias 
$b(k)=P_{\kappa \delta}(k)/P_{\delta \delta}(k)$ is a $k$-dependent bias. 
Four colored solid lines are 
corresponding different number density of halos fields.}
\label{fig:recon_noise}
\end{figure}
%==============================================================
\begin{figure}[tbp]
\begin{center}
\includegraphics[width=0.48\textwidth]{../../eps/recon_CC_2D.eps}
\includegraphics[width=0.48\textwidth]{../../eps/recon_bias_2D.eps}
\includegraphics[width=0.48\textwidth]{../../eps/recon_noise_b2Pd_2D.eps}
\end{center}
\vspace{-0.7cm}
\caption{We plot the 2D correlation coefficient, bias factor and noise 
of reconstructed $\kappa$ field from top to bottom, respectively. Here the 
definition of correlation coefficient and bias factor are same with
Fig. \ref{fig:recon_CC} and Fig. \ref{fig:recon_bias}, and the noise 
 in the bottom panel is defined as
 $(P_{\kappa}-b^2P_{\delta})/b^2P_{\delta}$. All three panels are
  reconstructed
 from halo fields with number density of $0.0024\ h^{3}\mathrm{Mpc}^{-3}$.
   }
\label{fig:recon_2D}
\end{figure}