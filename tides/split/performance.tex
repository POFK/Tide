%We have discussed the tidal reconstruction with local dark matter 
%density field with 2D and 3D method \citep{2015:zhu}. 
%In practice, however, we can only observe galaxies instead of dark matter
% density fields. And the distribution of halo is a biased tracer of the
%  underlying dark matter density. Therefore, the reconstruction in view of
% halo is even more relevant to the observation.\\

In this section we perform reconstruction in simulated halos fields,
following almost the same process in previous papers. 
To better understand the discrepancy between reconstruction in dark matter and
halos, we discuss the bias and noise of halos, which tell us how to decrease
the halos noise in reconstruction. And then, we also use the 
cross-correlation coefficient to quantify the result of reconstruction 
and discuss the bias and noise of it.\\

%==============================================================
\subsection{Simulation}
{\color{red}(*)}We run N-body simulations using the $\mathrm{CUBEP^3M}$ code with $2048^3$
 dark matter particles in a box of side length $L=1.2\ \mathrm{Gpc}/h$.
  We have
 adopted the following set of cosmological parameter values:
 $\Omega_{b}=0.049$, $\Omega_{c}=0.259$, $h=0.678$, 
 $A_s=2.139\times 10^{-9}$ and $n_s=0.968$.
 Ten runs with independent initial conditions were performed to provide
  better statistics. In the following calculations we use outputs 
  at $z=0$. \\
\subsection{\label{property_h}Properties of halos fields}
In this subsection, we choose three different number-density
($0.0048\ h^{3}\mathrm{Mpc}^{-3}$, 
$0.0024\ h^{3}\mathrm{Mpc}^{-3}$ and $0.0012\ h^{3}\mathrm{Mpc}^{-3}$)
 halos fields to show the properties of halos fields.
{\color{red} here, describe how to get these halos fields with different 
number density and the mass of each halos field. And it may be added in subsection of Simulation} \\

%we show the bias (without shot noise)
%power spectra of simulated
%halos fields $P_{hh}$.
% To understand better, we also show the bias (without shot noise) 
% of halos field 
%$b(k)=P_{\delta h}(k)/P_{\delta\delta}(k)$, and the noise power 
%$P_n(k)/b^{2}(k)=P_{hh}(k)/b^2(k)-P_{\delta \delta}(k)$, 
%defined via the stochasticity between halos and the dark matter.

%============halos property==Figure added here==========================
\begin{figure}[tbp]
\begin{center}
\includegraphics[width=0.48\textwidth]{../../eps/Sim_halo_property_bias.eps}
\end{center}
%\vspace{-0.7cm}
\caption{Halo biases of different number-density halos fields.
The dashed, the dash-dotted and the dotted line shows the bias of halos
 fields with different number density. And linear biases (horizontal solid
  lines) are  obtained by averaging the first 6 k-bins 
  ($k< 0.04\ h/\mathrm{Mpc}$) for each line. }
\label{fig:Sim_bias}
\end{figure}

\begin{figure}[tbp]
\begin{center}
\includegraphics[width=0.48\textwidth]{../../eps/Sim_halo_property_PS.eps}
\end{center}
%\vspace{-0.7cm}
\caption{Top panel: the auto-power spectrum of halo fields. The 
solid lines are the power spectrum of halo fields with number density of
$0.0048\ h^{3}\mathrm{Mpc}^{-3}$, $0.0036\ h^{3}\mathrm{Mpc}^{-3}$, 
$0.0024\ h^{3}\mathrm{Mpc}^{-3}$ and $0.0012\ h^{3}\mathrm{Mpc}^{-3}$,
 respectively.
The horizontal dot dash lines are corresponding shot noise $1/\bar{n}$ 
for every halo fields. Here $\bar{n}$ is the number density.
The black dashed line is the power
spectrum of dark matter field.
Bottom panel: same as the top panel, but the colored solid lines are the 
matter-halo cross power spectrum.}
\label{fig:Sim_PS}
\end{figure}



\begin{figure}[tbp]
\begin{center}
%\includegraphics[width=0.48\textwidth]{../../eps/Sim_halo_property_CC.eps}
\includegraphics[width=0.48\textwidth]{../../eps/Sim_halo_property_Pn.eps}
\end{center}
%\vspace{-0.7cm}
\caption{Top panel: the auto-power spectrum of halo fields. The 
solid lines are the power spectrum of halo fields with number density of
$0.0048\ h^{3}\mathrm{Mpc}^{-3}$, $0.0036\ h^{3}\mathrm{Mpc}^{-3}$, 
$0.0024\ h^{3}\mathrm{Mpc}^{-3}$ and $0.0012\ h^{3}\mathrm{Mpc}^{-3}$,
 respectively.
The horizontal dot dash lines are corresponding shot noise $1/\bar{n}$ 
for every halo fields. Here $\bar{n}$ is the number density.
The black dashed line is the power
spectrum of dark matter field.
Bottom panel: same as the top panel, but the colored solid lines are the 
matter-halo cross power spectrum.}
\label{fig:Sim_CC_Pn}
\end{figure}
%==============================================================
In Fig. \ref{fig:Sim_bias}, we show the bias factor  
 of simulated halos fields in three different number densities. 
 Here we define the halos
 bias factor as $b_{h}(k)=P_{\delta h}(k)/P_{\delta\delta}(k)$,
 which without the shot noise $1/\bar{n}$.
 Here, the $P_{\delta \delta}(k)$ and $P_{\delta h}$ are the auto-correlation 
 power
 spectrum of dark matter and the halo-matter cross-correlation power spectrum. 
 The dashed, the dash-dotted and the dotted lines show the bias of halos
 fields with number density $0.0012\ h^{3}\mathrm{Mpc}^{-3}$,
 $0.0024\ h^{3}\mathrm{Mpc}^{-3}$ and $0.0048\ h^{3}\mathrm{Mpc}^{-3}$,
 respectively. The solid line with corresponding color is the linear bias
 obtained by averaging the first 6 $k$-bins ($k< 0.04\ h/\mathrm{Mpc}$)
 of $b_{h}(k)$.
 On small scales, the bias is no longer a scale-independent factor. 
 In this paper, every point is averaged all ten simulations and the error
 is given by the bootstrap.\\
 
 The shot noise and stochasticity of halos are always an important and
  complicated topic in relevant studies {\color{blue} cite here}.
  A batter understanding about the shot noise and 
 stochasticity is helpful to decrease its influence in our reconstruction
 (in this paper, a wiener filter is applied to suppress the noise in halos
 fields, which we discuss this further below).
 The shot noise, assumed roughly to $1/\bar{n}$ here, can be subtract from
 the halos auto-cross power spectrum.
 {\color{blue} (Although some papers suggest the shot-noise subtraction is not safely.. may cite here?)} 
 And note that the matter power spectrum 
 dose not require shot noise subtraction because of the large number of 
 dark matter particles. In Fig. \ref{fig:Sim_PS}, we plot the power spectra
 of three different number-density halos fields. To compare these power
 spectra with dark matter power spectra, halos 
 shot-noise subtracted auto-correlation
 power spectra is divided by square of the bias factor 
   $P_{hh}(k)/b^2(k)$ (here, the bias we use is the scale-dependent bias and
   see Fig. \ref{fig:Sim_bias}). 
 Here, the black solid line is the matter power spectrum and coloured 
 lines indicate the halos auto-correlation power spectra of 
 different number density
 as in Fig. \ref{fig:Sim_bias}. We find, on large scales, different power
 spectra are almost have a same value. Which suggest 
  even with only shot-noise subtraction, the halos power spectra
 can trace the dark matter power well (less relative error) on large scales
 in choosed number density and halo mass. However, on 
 small scale, the halos power spectra are always above the dark matter power
 even applied the shot-noise subtraction, and higher number density with
 larger difference. So if we want to quantify the noise
 of halos fields, consider the
 shot noise only is not enough, especially on small scale.\\
 
 Understanding everything about the shot noise and the stochasticity is too
 complex and difficult, but it worth the effort. 
 However, we do not need to do it in our reconstruction. 
 Here, we define the noise power of halos field
\begin{equation}
P_{n}(k)=P_{hh}(k)-b^{2}_{h}(k)P_{\delta\delta}(k),
\label{eq:noise_halo}
\end{equation}
in which both the shot noise and the stochasticity have contribution.
By this equation, we can calculate the noise of halos from simulations.
In Fig. \ref{fig:Sim_CC_Pn}, we plot the $P_{n}(k)/b_{h}^{2}(k)$ 
to compare with
the dark matter power spectrum (the black solid line).
Identical to \ref{fig:Sim_PS}, we use the dashed, the dash-dotted 
and the dotted line show the noise power of different number density.
Three thin coloured solid lines indicate the shot noise of corresponding number density, which have been divided by $b^2_{h}(k)$ too. 
In this way, we can compare the shot noise $1/\bar{n}$, the noise of 
halos $P_{n}(k)$ and the power of dark matter $P_{\delta\delta}(k)$ directly.
On the linear scales ($k<0.1\ h/\mathrm{Mpc}$), all the noise lines is over
the corresponding shot noise line, which is because of stochasticity.
And comparing to the shot noise, the noise power $P_{n}$ is almost have 
the same order, which meaning the noise is dominated by stochasticity
on large scale in choosed number density and halo mass.
 In nonlinear region, the nonlinear
evolution involve much complex effects, we do not discuss it here.
The correlation coefficient of halos and dark matter,
which defined as $r_{\mathrm{h}}=P_{\delta h}/\sqrt{P_{\delta\delta}P_{hh}}$, is always
used to quantify the relation between the distribution of halo and matter.
 From Eq. \ref{eq:noise_halo} and the definition of $b_{h}(k)$ we can prove 
$\frac{P_{n}/b_{h}^2}{P_{\delta\delta}}=\frac{1}{r_{\mathrm{h}}^2}-1$,
so we do not show the correlation coefficient here.
\\

By these discussion about the bias, the power and the noise of halos
fields, we hope to find the differences between the 
reconstruction in halos and dark matter, which we discuss in subsection
 \ref{perf_of_recon}.
%==============================================================
%\begin{figure}[tbp]
%\begin{center}
%\includegraphics[width=0.48\textwidth]{../../eps/Sim_bias_errorbar.eps}
%\end{center}
%%\vspace{-0.7cm}
%\caption{Bias of simulated halo fields. Here we define the bias as 
%$b=\frac{P_{\delta h}}{P_{\delta}}$ and obtain a constant bias by
%averaging the first 6 $k$-bins where $k<0.04\ h/\mathrm{Mpc}$. The constant
%bias is $1.1137$, $0.9704$, $0.8986$ and $0.8456$ for the halo fields with
%number density 
%$0.0012\ (h/\mathrm{Mpc})^3$, $0.0024\ (h/\mathrm{Mpc})^3$, 
%$0.0036\ (h/\mathrm{Mpc})^3$ and $0.0048\ (h/\mathrm{Mpc})^3$, respectively.}
%\label{fig:Sim_bias}
%\end{figure}
%==============================================================
%\begin{figure}[tbp]
%\begin{center}
%\includegraphics[width=0.48\textwidth]{../../eps/Sim_Pn_errorbar.eps}
%\includegraphics[width=0.48\textwidth]{../../eps/Sim_noise_errorbar.eps}
%\end{center}
%%\vspace{-0.7cm}
%\caption{Noise of halo fields for $4$ number density. The top panel we show
%$P_{h}-b^2P_{\delta}$, here $b$ is the $k$-dependent bias 
%$P_{\delta h}/P_{h}$. The correlative horizontal dot dash lines are the 
%shot noise $1/\bar{n}$ of different number density field.
%The bottom panel is inverse signal to noise ratio 
%$(P_{h}-b^2P_{\delta})/b^2P_{\delta}$.}
%\label{fig:Sim_noise}
%\end{figure}
%==============================================================
%copyed from zhu,maybe add something by YuYu.
%==============================================================
\subsection{\label{perf_of_recon}Performance of reconstruction }
The performance of reconstruction in halos mainly following the previous works \citep{2012arXiv1202.5804P,2016PhRvD..93j3504Z}. 
However, there are some changes because
of the differences between the dark matter and halo fields. 
Note that because we are using the halos fields to trace dark matter 
distribution, the $\delta(x)$ used here is the halos density field 
({\color{red} explain how to get halos density field in Simulation or it should be another name?})
which have deconvolved bias $b_{h}$.
The algorithm of the reconstruction in halos is as follows.\\
%============================================
(i). Applying a Gaussian smoothing kernel to $\delta(x)$:
\begin{equation}
\bar{\delta}(\mathbf{x})=\int d^{3}x'\mathbf{S}(\mathbf{x}-\mathbf{x}')\delta	(\mathbf{x}'),
\label{equ:smooth kernel}
\end{equation}
where $\mathbf{S}(r)=\mathrm{e}^{-r^{2}/2R^{2}}$ and $R$ is the smoothing
scale. Here, we calculated a lots
of different smoothing scales and find we can obtain the optimal 
reconstruction result with the smoothing scale $R=1.0\ \mathrm{Mpc}/h$. 
So we select $R=1.0\ \mathrm{Mpc}/h$ as the smoothing scale in this paper. 
By this step, we get a smoothed
field $\bar{\delta}_{h}(x)$ which {\color{blue}is considered as} the
cosmic density field and have been suppressed the vary small-scale 
fluctuations.\\
{\color{red} discuss why different number-density halos reconstruction
with the same smoothing scale in Sec. Discussion.}\\
%============================================
(ii). Convolving the smoothed density field 
$\bar{\delta}_{h}(x)$ with a wiener filter 
{\color{red} (about the Wiener and $\delta^{w_{i}}$ should be discussed in 
Sec. Discussion: why we separate the Wiener from filter $w_i$.)}
\begin{equation}
W_{h}(k)=\frac{P_{hh}(k)-P_{n}(k)}{P_{hh}(k)}
=\frac{b_{h}^2(k)P_{\delta\delta}(k)}{P_{hh}(k)},
\label{eq:nbar_filter}
\end{equation}
where $P_{hh}(k)$ is the without shot-noise subtracted halos power spectrum,
$P_{\delta\delta}(k)$ the dark matter power spectrum, $b_{h}(k)$
the bias factor of halos and $P_{n}(k)$ the noise power from 
Eq. \ref{eq:noise_halo}.\\
The Wiener filter is help to suppress the noise of halos fields if we know
the signal and noise accurately. However, in real data, there is no 
approach to distinguish the signal and noise exactly, 
but we can in simulation. 
So the Wiener filter should be calculated in simulations.\\
The halos bias factor $b_h(k)$ used here is a scale-dependent bias.
However, a linear bias factor is usually believed more credible and can
be obtained from observation. In the Sec. \ref{sec:discussion}
{\color{red} here cite the figure}
, we will show that it is no influence on the reconstruction whether 
the scale-dependent (the curved lines in Fig. \ref{fig:Sim_bias}) or 
the scale-independent (the horizontal solid lines in Fig. \ref{fig:Sim_bias}) 
bias is used here.
\\
%=====
 After applying Eq. \eqref{eq:nbar_filter}, we
 suppress the noise in halo field (both the shot noise and stochasticity)
and obtain a smoothed and filtered halo density field $\tilde{\delta}(x)$
 to reconstruction.
\\
%============================================
(iii).{\color{red}*}
 Then, we convolve the smoothed and filtered density field 
$\tilde{\delta}(x)$ with $w_i$ (from Eq. \ref{equ:w})
 and calculate $\gamma_1$, $\gamma_2$, $\gamma_x$,
$\gamma_y$ and $\gamma_z$ by 
Eqs. \eqref{equ:gamma}\eqref{equ:w}\eqref{equ:Q}. And by
Eq. \eqref{equ:kappa3D} we obtain 3D convergence field 
$\kappa(\bm{k})$.\\
%============================================
(iv).{\color{red}*}
 By Eq. (\ref{eq:kappa}), we correct the bias and suppress the noise
of the estimated convergence field $\hat{\kappa}$. Here,
 $b(k_{\perp},k_{\parallel})$ and $W(k_{\perp},k_{\parallel})$ 
 are calculated by averaging in all ten simulations.
 {\color{blue} need discuss how to get reconstructed large-scale $\delta_{L}$
 from $\hat{\kappa}$?}\\
 
 The tidal shear estimator is an optimal estimator  under the Gaussian
  assumption, which have discussed in  Ref. \citep{2016PhRvD..93j3504Z}.
 However, we do not apply a Gaussianization method here. This is because of 
 the following points.
 Firstly, different from the dark matter density field, halos counts have 
 not such high fluctuations on small scale. So it is not necessary to
 Gaussianize to suppress the high weights of the high density regions 
 because of quadratic estimator. Secondly, for halos-counts fields, there are
 many grids without halos, which makes the Gaussianization method 
 lead into some other problems.  
 \\
%===================hereafter: result ====================================
