\section{Bias-weighting reconstruction}
We can combine the different mass bins of halos field 
with bias weighting by the following equation:
\begin{equation}
\delta_{w}=\frac{\sum b_{i}\delta_{i}}{\sum b_{i}/n},
\label{equ:bw}
\end{equation}
where $\delta_{i}$ is the $i$-th mass bin of halo fields, $b_{i}$ is the 
bias factor of the $i$-th mass bin and $n$ is the number of mass bins.
 By Eq. \eqref{equ:bw}, we are trying to
obtain a weighted halo field $\delta_{w}$ which can be a better tracer of 
dark matter density field.\\
To obtain a weighted field $\delta_{w}$, we sebsample the halo field by halos
mass. Here we separate the halos field into $4$ mass bins which have the same
number density.\\
%=============================================================================
\begin{figure}[tbp]
\begin{center}
\includegraphics[width=0.48\textwidth]{../../eps/bwSimCC0024M4bin.eps}
\end{center}
\vspace{-0.7cm}
\caption{The correlation coefficient of simulated halo mass bins 
and dark matter density
field. We separate the $0.0024\ (h/\mathrm{Mpc})^3$ halos fields into $4$ 
mass bins. From the highest to lowest mass we obtain mass bins: 
$m1$, $m2$, $m3$ and $m4$ and all of them with a same number density
$0.0006\ (h/\mathrm{Mpc})^3$. The blue solid line is the heaviest mass bin
 $m1$ and the green dashed line is the origin halo fields with number density
 $0.0024\ (h/\mathrm{Mpc})^3$}
\label{fig:bw_CC}
\end{figure}
%=============================================================================
\begin{figure}[tbp]
\begin{center}
\includegraphics[width=0.48\textwidth]{../../eps/bwSimNoiseE2Pd0024M4bin.eps}
\end{center}
\vspace{-0.7cm}
\caption{Same as Fig. \ref{fig:bw_CC} but we plot 
the inverse signal to noise ratio (N/S) of different mass bins. 
The colored lines are the (N/S) of different mass bins and the black solid 
line  indicates the origin halo fields with number density
 $0.0024\ (h/\mathrm{Mpc})^3$.
}
\label{fig:bw_noise}
\end{figure}
%=============================================================================
\begin{figure}[tbp]
\begin{center}
\includegraphics[width=0.48\textwidth]{../../eps/bwSimNoisePn0024M4bin.eps}
\includegraphics[width=0.48\textwidth]{../../eps/bwSimNoisePn0003M10bin.eps}
\end{center}
\vspace{-0.7cm}
\caption{The noise of halos mass bins and original halos fields.
In the top and the bottom panel, we separate $0.0024\ (h/\mathrm{Mpc})^3$ 
and $0.0003\ (h/\mathrm{Mpc})^3$ halos fields into $4$ and $10$ mass bins, 
respectively. The colored lines indicate different mass bins from the 
heaviest to the lightest mass bin (from $m1$ to $m4$ or from $\mathrm{bin1}$
 to $\mathrm{bin10}$) and the black solid line 
 indicates the original halos field 
($0.0024\ (h/\mathrm{Mpc})^3$ and $0.0003\ (h/\mathrm{Mpc})^3$).}
\label{fig:bw_Pnoise}
\end{figure}
%=============================================================================
\begin{figure}[tbp]
\begin{center}
\includegraphics[width=0.48\textwidth]{../../eps/bwReconCC0024M4bin.eps}
\end{center}
\vspace{-0.7cm}
\caption{Correlation coefficient of reconstruction from bias-weighted halos 
field and matter density field. Here we choose the number density of 
$0.0024\ (h/\mathrm{Mpc})^3$ and combine with $4$ mass bins.}
\label{fig:bw_recon_CC}
\end{figure}