The large-scale structure contains a wealth of information 
about our Universe, including cosmic acceleration, neutrino masses
and inflation. By measuring large-scale structure we can attempt to
find the answers to such fundamental questions as what the initial
conditions for the Universe are and what the future of the Universe will be.
% The cosmic acceleration, neutrino masses, early universe models
%and other properties of the Universe can be inferred with current or 
%upcoming surveys. 
However, the nonlinear evolution and many observational
problems make it difficult to extract cosmological information. In our 
previous studies \citep{2012:pen,2015:zhu}, the gravitational coupling
between the long wavelength tidal field and small scale density fluctuations
is used to reconstruct the large-scale density field, which provides a new
method (cosmic tidal reconstruction) to extract the large-scale information from local observation.\\
%Cosmic tidal
%reconstruction allows us to use small-scale filamentary structures to 
%solve the large-scale tidal shear and gravitational potential.\\

The basic idea of the cosmic tidal reconstruction is that the local
anisotropic structure includes the large scale information because of the
tidal interactions in different scale modes. Each independent measurement 
of the small scale power spectrum gives some information about the power
spectrum on large scales. So we can use cheaper local observation to 
estimate the long wavelength tidal field by cosmic tidal reconstruction,
 which can be used to recover lost 21cm modes and reduce sample 
 variance \citep{2012:pen}. We have tested the reconstruction with simulated 
dark matter density fields \citep{2015:zhu} and have developed a 
three-dimensional (3D) method which obtains even better result (in
preparation). However, the reconstruction with dark matter density field
can not help us to extend this method to actual data in surveys and more 
effects should be done.\\

Galaxy survey is one of the leading methods to measure the clustering
of dark matter and has been

Dark matter halos, which host observable galaxies and galaxy clusters, are
biased tracers of the underlying dark matter density field. The relation
between the galaxy and dark matter halos be described by the halo model
\citep{Halo_model1:2001ApJ,Halo_model2:2000MNRAS,Halo_model3:2000MNRAS}.
Therefore, understanding tidal reconstruction in view of halo field is
important for us to extract information from galaxy surveys, which we will
 study in our next work. In this
paper, we extend 3D tidal reconstruction in view of halos and try to
reconstruction with a bias-weighting halos field to get a higher correlation
between reconstructed and original density field.(*)\\

In our previous studies \citep{2012:pen,2015:zhu}, we applied a
 Gaussianization technique to suppress the large weight of the high density
 regions because of the quadratic estimator. However, for the halos fields, 
 there is not such a high fluctuation. So we have a good reconstruction even
 without Gaussianization and which gives us a simpler and clearer physical
 images. Furthermore, without Gaussianization, we can estimator the bias of
 tidal reconstruction easily and correct it, which is import for our 
 reconstruction. \\
  
The shot noise, which due to the discrete sampling of halos, is another
 problem that should be solved. Generally speaking, we assume the shot
 noise is the inverse of the number density $\bar{n}$. So for the dark matter
 density field, the shot noise is too small to ignore. However, the 
 influence of the shot noise limits the result of reconstruction in view of 
 hales. Here, we apply a wiener filter to suppress the shot noise, which
 we will introduce later.\\
 
Galaxy survey has been one of the leading methods to measure the clustering
of dark matter. There have been a lot of efforts in developing optimal 
weighting in galaxy survey \citep{OW1,OW2,OW3,OW4,OW5}. It is difficult to 
get the mass information in galaxy survey. However we can apply a biased
weighting, that bias can be given by galaxy luminosity, to improve the 
signal to noise ratio of the halo fields. In this paper, we are not try to
find a optimal bias weighting, but the optimal weighting may be needed in
some case.\\

(*)This paper is organized as follows. In Section \ref{sec:review}, 
we review the
 reconstruction technique. In Section \ref{sec:performance},
  we study the performance of
  reconstruction in halo density fields from $N$-body simulations. 
 In Section \ref{sec:cross_term}, we study how much improvement can 
 get from bias weighted halo fields.\\
 