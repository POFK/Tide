The large-scale structure contains a wealth of information about our Universe.
By measuring large-scale structure we can attempt to
find the answers to such fundamental questions as what the initial
conditions of the Universe are and what its future will be.
{\color{red}There have been a lot of efforts in measuring the large-scale
structure of our Universe (e.g. SDSS,).} 
However, the mode coupling because of nonlinear evolution 
{\color{blue} (here add citation)} and many other observational
problems {\color{red}(such as RSD)} make it difficult to extract cosmological information. 
Additionally, because of the limited volume of surveys, uncertainties 
due to sample variance is also difficult to overcome.
%==================================================================

Dark matter halos, which host observable galaxies and galaxy clusters, 
are the fundamental nonlinear units of cosmic structure. On large scales,
 dark matter halos can be considered as biased tracers 
of the underlying dark matter distribution \citep{1984ApJ...284L...9K}.
Not only the local matter distribution but also
the large-scale environment can affect the shapes, the abundance, 
the accretion, the bias, and the power spectrum of halos
\citep{2016ApJ...825...49C,2012JCAP...03..004S,1996MNRAS.282..347M,2015ApJ...807...37S,2015JCAP...09..028C,2014PhRvD..90j3530L,2014JCAP...05..048C,2013PhRvD..87l3504T}.
{\color{blue}
So the coupling between the large-scale environment and small-scale 
{\color{red}density fluctuation} may be used to extract more information about 
the large-scale structure, which is helpful to suppress the sample variance
of large-scale modes.}
In recent studies, this effect is used to get the 
information of very long-wavelength modes
outside the survey volume by considering the large-scale environment
contribute to the mean density fluctuation in the survey
(or "separate universe")
\citep{2016JCAP...09..007B,2016arXiv160901701C,2016PhRvD..93f3507L,2016JCAP...02..018L, 2015JCAP...08..042W,2003ApJ...585...34M}.
Nevertheless, the isotropic distortions on small scales may derived from
not only the large-scale environment but also some other processes.
In our previous stydies \citep{2012arXiv1202.5804P,2016PhRvD..93j3504Z},
the long-wavelength tidal field is reconstructed by quantifying the local
anisotropy of small-scale density statistics and even a better result
obtained by a developed 3-D method \citep{tides3d:Hu}.
%If we only consider the position of halos, the large-scale density field
%is also can be reconstructed by halos field.
%==================================================================

In this paper, we investigate the cosmic tidal reconstruction in view
of halos.
Different with dark matter density field, reconstruction with halos field has
many advantages. In the dark matter case, we need to take a logarithmic 
transform or map the density fluctuations into a Gaussian distribution to
Gaussianize the smoothed density field or suppress the high weights 
of the high density regions because of quadratic estimator
\citep{2012arXiv1202.5804P,2016PhRvD..93j3504Z,tides3d:Hu}.
However, for halo counts, there is not such problems. Without Gaussianization,
we can quantify the bias of reconstruction easier and get a simpler and 
clearer physical picture, which we will discuss in \ref{sec:performance}. 
Additionally, since most observed tracers of large-scale structure, 
such as galaxies, reside in halos, the statistics of halos determine
 galaxies on large scale. So the study of halos reconstruction is helpful
 for studying the reconstruction with galaxy surveys, 
 which we will study in the future.
 
%==================================================================
% halo model and properties; biased tracer; what should we be cautious to in reconstruction.
On large scale, the relation between halo and dark matter can be modeled by
a bias factor $b(k)$ and numerous studies have been made about it
\citep{1986ApJ...304...15B,1996MNRAS.282..347M,1999MNRAS.308..119S,2001MNRAS.323....1S,2004MNRAS.355..129S,2011MNRAS.415..383M,2010ApJ...724..878T,2011JCAP...10..031B,2016MNRAS.458.1510S}.
As a biased tracer of dark matter, halos with different mass trace dark matter
in different degrees. 
{\color{blue}So can we get a better result in reconstruction by 
dealing different halos with different ways is an interesting thing.}
In addition, the shot noise, which due to the discrete sampling of halos,
always assumed to the inverse of the number density $1/\bar{n}$, is another
 problem that should be solved. 
 Here, we apply a wiener filter to suppress the shot noise, which
 we will introduce later.\\
%==================================================================

% The relationinvestigate
%between the galaxy and dark matter halos be described by the halo model
%\citep{Halo_model1:2001ApJ,Halo_model2:2000MNRAS,Halo_model3:2000MNRAS}.
%Therefore, understanding tidal reconstruction in view of halo field is
%important for us to extract information from galaxy surveys, which we will
% study in our next work. In this
%paper, we extend 3D tidal reconstruction in view of halos and try to
%reconstruction with a bias-weighting halos field to get a higher correlation
%between reconstructed and original density field.(*)
%==================================================================
%
% In our 
%previous studies \citep{2012:pen,2015:zhu}, the gravitational coupling
%between the long wavelength tidal field and small scale density fluctuations
%can be used to reconstruct the large-scale density field, which provides a new
%method (cosmic tidal reconstruction) to extract the large-scale information from small-scale observables. This provides many independent samples to reduce the sample variance of the measurement of large-scale structure. In addition,
%a developed 3D method obtains even a better result \citep{tides3d}. \\
%%==================================================================

%The basic idea of the cosmic tidal reconstruction is that the local
%anisotropic structure includes the large scale information because of the
%tidal interactions in different scale modes. Each independent measurement 
%of the small scale power spectrum gives some information about the power
%spectrum on large scales. So we can use cheaper local observation to 
%estimate the long wavelength tidal field by cosmic tidal reconstruction,
% which can be used to recover lost 21cm modes and reduce sample 
% variance \citep{2012:pen}. We have tested the reconstruction with simulated 
%dark matter density fields \citep{2015:zhu} and have developed a 
%three-dimensional (3D) method which obtains even better result (in
%preparation). However, the reconstruction with dark matter density field
%can not help us to extend this method to actual data in surveys and more 
%effects should be done.\\

%Galaxy survey is one of the leading methods to measure the clustering
%of dark matter and has been



%  
%The shot noise, which due to the discrete sampling of halos, is another
% problem that should be solved. Generally speaking, we assume the shot
% noise is the inverse of the number density $1/\bar{n}$. 
% So for the dark matter
% density field, the shot noise is too small to ignore. However, the 
% influence of the shot noise limits the result of reconstruction in view of 
% hales. Here, we apply a wiener filter to suppress the shot noise, which
% we will introduce later.\\
 
%Galaxy survey has been one of the leading methods to measure the clustering
%of dark matter. There have been a lot of efforts in developing optimal 
%weighting in galaxy survey \citep{OW1,OW2,OW3,OW4,OW5}. It is difficult to 
%get the mass information in galaxy survey. However we can apply a biased
%weighting, that bias can be given by galaxy luminosity, to improve the 
%signal to noise ratio of the halo fields. In this paper, we are not try to
%find a optimal bias weighting, but the optimal weighting may be needed in
%some case.\\

(*)This paper is organized as follows. In Section \ref{sec:review}, 
we review the
 reconstruction technique. In Section \ref{sec:performance},
  we study the performance of
  reconstruction in halo density fields from $N$-body simulations. 
 In Section \ref{sec:cross_term}, we study how much improvement can 
 get from bias weighted halo fields.\\
 
